%% 
%% The Project Gutenberg EBook of The Meditations of the Emperor Marcus
%% Aurelius Antoninus, by Marcus Aurelius
%% 
%% This eBook is for the use of anyone anywhere in the United States and most
%% other parts of the world at no cost and with almost no restrictions
%% whatsoever.  You may copy it, give it away or re-use it under the terms of
%% the Project Gutenberg License included with this eBook or online at
%% www.gutenberg.org.  If you are not located in the United States, you'll have
%% to check the laws of the country where you are located before using this ebook.
%% 
%% Title: The Meditations of the Emperor Marcus Aurelius Antoninus
%%        A new rendering based on the Foulis translation of 1742
%% 
%% Author: Marcus Aurelius
%% 
%% Translator: George Chrystal
%% 
%% Release Date: August 9, 2017 [EBook #55317]
%% 
%% Language: English
%% 
%% 
%% *** START OF THIS PROJECT GUTENBERG EBOOK MEDITATIONS OF MARCUS AURELIUS ANTONINUS ***
%% 

\documentclass{book}

\usepackage{fontspec}
\setmainfont[Numbers=OldStyle]{TeX Gyre Schola}

\usepackage{fmtcount}
\usepackage[papersize={5.5in,8.5in}]{geometry}
\usepackage{hyperref}

\usepackage[all]{nowidow}

\def\numberline#1{#1\hfill}

\makeatletter
\renewcommand\@pnumwidth{4.55em}
\makeatother

\renewcommand\chaptername{Book}
\renewcommand{\thechapter}{\Roman{chapter}}

\newcommand\terminus[1]{\vspace{2em}\emph{#1} \\[2em] \begin{center}End of the \ordinalstring{chapter} book\end{center}}

\newcommand\simpletitle{The Meditations \\ of the \\ Emperor \\ Marcus Aurelius Antoninus}

\title{\simpletitle \\[1em] {\small A New Rendering based on the Foulis Translation of 1742}}
%\title{The Meditations \\ of the \\ Emperor \\ Marcus Aurelius Antoninus \\[1em] {\small A New Rendering based on the Foulis Translation of 1742}}

\author{Marcus Aurelius Antoninus \\[2em] Text by George W. Chrystal \\ {\small Warner Exhibitioner of Balliol College, Oxford} \\[2em] 1902}

\date{}

\begin{document}

\setlength{\baselineskip}{1.2\baselineskip}

\frontmatter

\thispagestyle{empty}
\hspace{0pt}
\pagebreak

\thispagestyle{empty}
\hspace{0pt}
\pagebreak

\thispagestyle{empty}
\hspace{0pt}
\pagebreak

\thispagestyle{empty}
\hspace{0pt}
\pagebreak

\thispagestyle{empty}
\hspace{0pt}
\vfill
\begin{center}
  \Large
  \simpletitle
\end{center}
\vfill
\pagebreak

\thispagestyle{empty}
\hspace{0pt}
\vfill

Produced by E.H.N for Project Gutenberg.

Typeset by Tommy M. McGuire using \LaTeX $2_\epsilon$.

The font is \TeX\ Gyre Schola.

\pagebreak
\thispagestyle{empty}

\maketitle

\pagebreak

\hspace{0pt}
\vfill

\emph{\href{http://www.gutenberg.org/ebooks/55317}{http://www.gutenberg.org/ebooks/55317}}

\emph{This eBook is for the use of anyone anywhere in the United States and
most other parts of the world at no cost and with almost no
restrictions whatsoever. You may copy it, give it away or re-use it
under the terms of the Project Gutenberg License included with this
eBook or online at www.gutenberg.org. If you are not located in the
United States, you'll have to check the laws of the country where you
are located before using this ebook.}

\hspace{0pt}

\addtocontents{toc}{\textbf{Book}~\hfill\textbf{Page}\par}
\tableofcontents

\mainmatter

\chapter[I learned from my grandfather...]{}

1. \textsc{I learned from my grandfather,} Verus, to use good manners,
and to put restraint on anger.

2. In the famous memory of my father I had a pattern of modesty and
manliness.

3. Of my mother I learned to be pious and generous; to keep myself not
only from evil deeds, but even from evil thoughts; and to live with a
simplicity which is far from customary among the rich.

4. I owe it to my great-grandfather that I did not attend public
lectures and discussions, but had good and able teachers at home; and I
owe him also the knowledge that for things of this nature a man should
count no expense too great.

5. My tutor taught me not to favour either green or blue at the
chariot races, nor, in the contests of gladiators, to be a supporter
either of light or heavy armed. He taught me also to endure labour;
not to need many things; to serve myself without troubling others; not
to intermeddle in the affairs of others, and not easily to listen to
slanders against them.

\newpage

6. Of Diognetus I had the lesson not to busy myself about vain things;
not to credit the great professions of such as pretend to work
wonders, or of sorcerers about their charms, and their expelling of
Demons and the like; not to keep quails (for fighting or divination),
nor to run after such things; to suffer freedom of speech in others,
and to apply myself heartily to philosophy. Him also I must thank for
my hearing first Bacchius, then Tandasis and Marcianus; that I wrote
dialogues in my youth, and took a liking to the philosopher's pallet
and skins, and to the other things which, by the Grecian discipline,
belong to that profession.

7. To Rusticus I owe my first apprehensions that my nature needed
reform and cure; and that I did not fall into the ambition of the
common Sophists, either by composing speculative writings or by
declaiming harangues of exhortation in public; further, that I never
strove to be admired by ostentation of great patience in an ascetic
life, or by display of activity and application; that I gave over the
study of rhetoric, poetry, and the graces of language; and that I did
not pace my house in my senatorial robes, or practise any similar
affectation. I observed also the simplicity of style in his letters,
particularly in that which he wrote to my mother from Sinuessa. I
learned from him to be easily appeased, and to be readily reconciled
with those who had displeased me or given cause of offence, so soon as
they inclined to make their peace; to read with care; not to rest
satisfied with a slight and superficial knowledge; nor quickly to
assent to great talkers. I have him to thank that I met with the
discourses of Epictetus, which he furnished me from his own library.

\newpage

8. From Apollonius I learned true liberty, and tenacity of purpose; to
regard nothing else, even in the smallest degree, but reason always;
and always to remain unaltered in the agonies of pain, in the losses
of children, or in long diseases. He afforded me a living example of
how the same man can, upon occasion, be most yielding and most
inflexible. He was patient in exposition; and, as might well be seen,
esteemed his fine skill and ability in teaching others the principles
of philosophy as the least of his endowments. It was from him that I
learned how to receive from friends what are thought favours without
seeming humbled by the giver or insensible to the gift.

9. Sextus was my pattern of a benign temper, and his family the model
of a household governed by true paternal affection, and a steadfast
purpose of living according to nature. Here I could learn to be grave
without affectation, to observe sagaciously the several dispositions
and inclinations of my friends, to tolerate the ignorant and those who
follow current opinions without examination. His conversation showed
how a man may accommodate himself to all men and to all companies; for
though companionship with him was sweeter and more pleasing than any
sort of flattery, yet he was at the same time highly respected and
reverenced. No man was ever more happy than he in comprehending,
finding out, and arranging in exact order the great maxims necessary
for the conduct of life. His example taught me to suppress even the
least appearance of anger or any other passion; but still, with all
this perfect tranquillity, to possess the tenderest and most
affectionate heart; to be apt to approve others yet without noise; to
have much learning and little ostentation.

10. I learned from Alexander the Grammarian to avoid censuring others,
to refrain from flouting them for a barbarism, solecism, or any false
pronunciation. Rather was I dexterously to pronounce the words rightly
in my answer, confining approval or objection to the matter itself,
and avoiding discussion of the expression, or to use some other form
of courteous suggestion.

11. Fronto made me sensible how much of envy, deceit and hypocrisy
surrounds princes; and that generally those whom we account nobly born
have somehow less natural affection.

12. I learned from Alexander the Platonist not often nor without great
necessity to say, or write to any man in a letter, that I am not at
leisure; nor thus, under pretext of urgent affairs, to make a practice
of excusing myself from the duties which, according to our various
ties, we owe to those with whom we live.

13. Of Catulus I learned not to condemn any friend's expostulation
even though it were unjust, but to try to recall him to his former
disposition; to stint no praise in speaking of my masters, as is
recounted of Domitius and Athenodorus; and to love my children with
true affection.

14. Of Severus, my brother, I learned to love my kinsmen, to love
truth, to love justice. Through him I came to know Thrasea, Helvidius,
Cato, Dion, and Brutus. He gave me my first conception of a
Commonwealth founded upon equitable laws and administered with
equality of right; and of a Monarchy whose chief concern is the
freedom of its subjects. Of him I learned likewise a constant and
harmonious devotion to Philosophy; to be ready to do good, to be
generous with all my heart. He taught me to be of good hope and
trustful of the affection of my friends. I observed in him candour in
declaring what he condemned in the conduct of others; and so frank and
open was his behaviour, that his friends might easily see without the
trouble of conjecture what he liked or disliked.

15. The counsels of Maximus taught me to command myself, to judge
clearly, to be of good courage in sickness and other misfortunes, to
be moderate, gentle, yet serious in disposition, and to accomplish my
appointed task without repining. All men believed that he spoke as he
thought; and whatever he did, they knew it was done with good intent.
I never found him surprised or astonished at anything. He was never in
a hurry, never shrank from his purpose, was never at a loss or
dejected. He was no facile smiler, but neither was he passionate or
suspicious. He was ready to do good, to forgive, and to speak the
truth, and gave the impression of unperverted rectitude rather than of
a reformed character. No man could ever think himself despised by
Maximus, and no one ever ventured to think himself his superior. He
had also a good gift of humour.

16. I learned from my father gentleness and undeviating constancy in
judgments formed after due reflection; not to be puffed up with glory
as men understand it; to be laborious and assiduous. He taught me to
give ready hearing to any man who offered anything tending to the common
good; to mete out impartial justice to every one; to apprehend rightly
when severity and when clemency should be used; to abstain from all
impure lusts; and to use humanity towards all men. Thus he left his
friends at liberty to sup with him or not, to go abroad with him or
not, exactly as they inclined; and they found him still the same if
some urgent business had prevented them from obeying his commands. I
learned of him accuracy and patience in council, for he never quitted an
enquiry satisfied with first impressions. I observed his zeal to retain
his friends without being fickle or over fond; his contentment in every
condition; his cheerfulness; his forethought about very distant events;
his unostentatious attention to the smallest details; his restraint of
all popular applause and flattery.

Ever watchful of the needs of the Empire, a careful steward of the
public revenue, he was tolerant of the censure of others in affairs of
that kind. He was neither a superstitious worshipper of the Gods, nor
an ambitious pleaser of men, nor studious of popularity, but in all
things sober and steadfast, well skilled in what was honourable, never
affecting novelties. As to the things which make the ease of life, and
which fortune can supply in such abundance, he used them without pride,
and yet with all freedom: enjoyed them without affectation when they
were present, and when absent he found no want of them.

No man could call him sophist, buffoon, or pedant. He was a man of ripe
experience, a full man, one who could not be flattered, and who could
govern himself as well as others. I further observed that he honoured
all who were true philosophers, without upbraiding the rest, and without
being led astray by any.

His manners were easy, his conversation delightful, but not cloying. He
took regular but moderate care of his body, neither as one over fond of
life or of the adornment of his person, nor as one who despised these
things. Thus, through his own care, he seldom needed any medicines,
whether salves or potions. It was his special merit to yield without
envy to any who had acquired any special faculty, as either eloquence,
or learning in the Law, in ancient customs, or the like; and he aided
such men strenuously, so that every one of them might be regarded
and esteemed for his special excellence. He observed carefully the
ancient customs of his forefathers, and preserved, without appearance
of affectation, the ways of his native land. He was not fickle and
capricious, and loved not change of place or employment. After his
violent fits of headache he would return fresh and vigorous to his
wonted affairs.

Of secrets he had few, and these seldom, and such only as concerned
public matters. He displayed discretion and moderation in exhibiting
shows for the entertainment of the people, in his public works, in
largesses and the like; and in all those things he acted like one who
regarded only what was right and becoming in the things themselves,
and not the reputation that might follow after. He never bathed at
unseasonable hours, had no vanity in building, was never solicitous
either about his food or about the make or colour of his clothes, or
about the beauty of his servants. His dress came from Lorium--his
villa on the coast--and was of Lanuvian wool for the most part. It is
remembered how he used the tax-collector at Tusculum who asked his
pardon, and all his behaviour was of a piece with that. He was far from
being inhuman, or implacable, or violent; never doing anything with such
keenness that one could say he was sweating about it, in all things he
reasoned distinctly, as one at leisure, calmly, regularly, resolutely,
and consistently. A man might fairly apply that to him which is recorded
of Socrates: that he could both abstain from and enjoy these things,
in want whereof many show themselves weak, and, in the possession,
intemperate. To be strong in abstinence and temperate in enjoyment, to
be sober in both--these are qualities of a man of perfect and invincible
soul, as was shown in the sickness of Maximus.

17. To the Gods I owe it that I had good grandfathers and parents, a
good sister, good teachers, good servants, good kinsmen, and friends,
good almost all of them. I have to thank them that I never through
haste and rashness offended any of them; though my temper was such as
might have led me to it had occasion offered. But by their goodness no
such concurrence of circumstances happened as could discover my
weakness. I am further thankful that I was not longer brought up with my
grandfather's concubine, that I retained my modesty, and refrained even
longer than need have been from the pleasures of love.

To the Gods it is due that I lived under the government of such a
prince and father as could take from me all vain glory, and convince
me that it was not impossible for a prince to live in a court without
guards, gorgeous robes, torches, statues, or such pieces of state and
magnificence; but that he may reduce himself almost to the state of a
private man, and yet not become more mean or remiss in those public
affairs wherein power and authority are requisite. I thank the Gods that
I have had such a brother as by his disposition might stir me to take
care of myself, while at the same time he delighted me by his respect
and love. I thank them that my children neither wanted good natural
dispositions nor were deformed in body.

I owe it to their good guidance that I made no greater progress in
rhetoric and poetry, and in other studies which might have engrossed
my mind had I found myself successful in them. By the Gods' grace I
forestalled the wishes of those by whom I was brought up, in promoting
them to the dignities they seemed most to desire; and I did not put
them off with the hope that, since they were but young, I would do it
hereafter. I owe to the Gods that I ever knew Apollonius, Rusticus and
Maximus; that I have had occasion often and effectually to meditate with
myself and enquire what is truly the life according to Nature. And, as
far as lies within the dispensation of the Gods to give suggestion,
help, or inspiration, there is nothing to prevent my having already
realized that life. I have fallen short of it by my own fault, and
because I gave no heed to the inward monitions and almost direct
instructions of the Gods, to whom be thanks that my body hath so long
endured the stress of such a life as I have led. By their goodness I
never had to do with either Benedicta or Theodotus; and afterwards, when
I fell into some foolish passions, I was soon cured. I give thanks that,
having often been displeased with Rusticus, I never did anything to
him which afterwards I might have had occasion to repent; that, though
my mother was destined to die young, she lived with me all her latter
years; that, as often as I inclined to succour any who were either poor
or had fallen into some distress, I was never answered that there was
not ready money enough to do it, and that I myself never had need of the
like succour from another. I must be grateful, too, that I have such a
wife, so obedient, so loving, so ingenuous; that I had choice of fit and
able men to whom I might commit the education of my children.

I have received divine aids in dreams; as in particular, how I might
stay my spitting of blood and cure my vertigo; which good fortune
happily fell to me at Caieta. The Gods watched over me also when I first
applied myself to philosophy. For I fell not into the hands of any
Sophist, nor sat poring over many volumes, nor devoted myself to solving
syllogisms, or star-gazing. That all these things should so happily fall
out there was great need both for the help of fortune and for the aid of
the Gods.

\terminus{In the country of the Quadi, by the Granua.}

\chapter[Say this to yourself...]{}

1. \textsc{Say this to yourself} in the morning: Today I shall have
to do with meddlers, with the ungrateful, with the insolent, with the
crafty, with the envious and the selfish. All these vices have beset
them, because they know not what is good and what is evil. But I have
considered the nature of the good, and found it beautiful: I have
beheld the nature of the bad, and found it ugly. I also understand the
nature of the evil-doer, and know that he is my brother, not because
he shares with me the same blood or the same seed, but because he is
a partaker of the same mind and of the same portion of immortality. I
therefore cannot be hurt by any of these, since none of them can involve
me in any baseness. I cannot be angry with my brother, or sever myself
from him, for we are made by nature for mutual assistance, like the
feet, the hands, the eyelids, the upper and lower rows of teeth. It is
against nature for men to oppose each other; and what else is anger and
aversion?

2. All that I am is either flesh, breath, or the ruling part. Cast
your books from you; distract yourself no more; for you have not the
right to do so. Like one at the point of death despise this flesh,
this corruptible bone and blood, this network texture of nerves,
veins, and arteries. Consider, too, what breath is--mere air, and
that always changing, expelled and inhaled again every moment. The
third is the ruling part. As to this, take heed, now that you are old,
that it remain no longer in servitude; that it be no more dragged
hither and thither like a puppet by every selfish impulse. Repine no
more at what fate now sends, nor dread what may befall you hereafter.

3. Whatever the Gods ordain is full of wise forethought. The workings
of chance are not apart from nature, and not without connexion and
intertexture with the designs of Providence. Providence is the source
of all things; and, besides, there is necessity, and the utility of
the Universe, of which you are a part. For, to every part of a being,
that is good which springs from the nature of the whole and tends to
its preservation. Now, the order of Nature is preserved in the changes
of elements, just as it is in the changes of things that are
compound. Let this suffice you, and be your creed unchangeable. Put
from you the thirst of books, that you may not die murmuring, but
meekly, and with true and heartfelt gratitude to the Gods.

4. Think of your long procrastination, and of the many opportunities
given you by the Gods, but left unused. Surely it is high time to
understand the Universe of which you are a part, and the Ruler of that
Universe, of whom you are an emanation; that a limit is set to your
days, which, if you use them not for your enlightenment, will depart,
as you yourself will, and return no more.

5. Hourly and earnestly strive, as a Roman and a man, to do what falls
to your hand with perfect unaffected dignity, with kindliness, freedom
and justice, and free your soul from every other imagination. This you
will accomplish if you perform each action as if it were your last,
without wilfulness, or any passionate aversion to what reason
approves; without hypocrisy or selfishness, or discontent with the
decrees of Providence. You see how few things it is necessary to
master in order that a man may live a smooth-flowing, God-fearing
life. For of him that holds to these principles the Gods require no
more.

6. Go on, go on, O my soul, to affront and dishonour thyself! The time
that remains to honour thyself will not be long. Short is the life of
every man; and thine is almost spent; spent, not honouring thyself,
but seeking thy happiness in the souls of other men.

7. Cares from without distract you: take leisure, then, to add some
good thing to your knowledge; have done with vacillation, and avoid
the other error. For triflers, too, are they who, by their activities,
weary themselves in life, and have no settled aim to which they may
direct, once and for all, their every desire and project.

8. Seldom are any found unhappy from not observing what is in the
minds of others. But such as observe not well the stirrings of their
own souls must of necessity be unhappy.

9. Remember always what the nature of the Universe is, what your own
nature is, and how these are related--the one to the other. Remember
what part your qualities are of the qualities of the whole, and that
no man can prevent you from speaking and acting always in accordance
with that nature of which you are a part.

10. In comparing crimes together, as, according to the common idea,
they may be compared, Theophrastus makes the true philosophical
distinction, that those committed from motives of pleasure are more
heinous than those which are due to passion. For he who is a prey to
passion is clearly turned away from reason by some spasm and
convulsion that takes him unawares. But he who sins from desire is
conquered by pleasure, and so seems more incontinent and more
effeminate in his vice. Justly then, and in a truly philosophical
spirit, he says that sin, for pleasure's sake, is more wicked than sin
which is due to pain. For the latter sinner was sinned against, and so
driven to passion by his wrongs, while the former set out to sin of
his own motion, and was led into ill-doing by his own lust.

11. Do every deed, speak every word, think every thought in the
knowledge that you may end your days any moment. To depart from men,
if there be really Gods, is nothing terrible. The Gods could bring no
evil thing upon you. And if there be no Gods, or if they have no
regard to human affairs, why should I desire to live in a world void
of Gods and without Providence? But Gods there are, and assuredly they
regard human affairs; and they have put it wholly in man's power that
he should not fall into what is truly evil. And of other things, had
any been bad, they would have made provision also that man should have
the power to avoid them altogether. For how can that make a man's life
worse which does not corrupt the man himself? Presiding Nature could
not in ignorance, or in knowledge impotent, have omitted to prevent or
rectify these things. She could not fail us so completely that, either
from want of power or want of skill, good and evil should happen
promiscuously to good men and to bad alike. Now death and life, glory
and reproach, pain and pleasure, riches and poverty--all these
happen equally to the good and to the bad. But, as they are neither
honourable nor shameful, they are therefore neither good nor evil.

12. It is the office of our rational power to apprehend how swiftly
all things vanish; how the corporeal forms are swallowed up in the
material world, and the memory of them in the tide of ages. Such are
all the things of sense, especially those which ensnare us with
pleasure or terrify us with pain, or those things which vanity
trumpets in our ears. How mean, how despicable, how sordid, how
perishable, how dead are they! What are they whose opinions and whose
voices bestow renown? What is it to die? Your mind can tell you that,
did a man think of it alone, and, by close consideration, strip it of
its ghastly trappings, he would no longer deem it anything but a work
of Nature. To dread a work of Nature is a childish thing, and this is,
indeed, not only Nature's work, but beneficial to her. Your reason
tells you how man reaches God, and through what part, and what is the
state of that part, when he has attained unto him.

13. Nothing, says the poet, is more miserable than to range over all
things, to spy into the depths of the earth, and search, by
conjecture, into the souls of those around us, yet not to perceive
that it is enough for a man to devote himself to that divinity which
is within him, and to pay it genuine worship. And this worship
consists in keeping it pure from every passion and folly, and from
repining at anything done by Gods or men. The work of the Gods is to
be reverenced for its excellence. The works of men should be dear for
the sake of the bond of kinship, or pitied, as we must pity them
sometimes, for their lack of the knowledge of good and evil. And men
are not less maimed by this defect than by their want of power to know
white from black.

14. Though you should live three thousand ears or as many myriads, yet
remember that no man loses any other life than that which now lives,
nor lives any other than that which he is now losing. The longest and
the shortest lives come to one effect. The present moment is the same
for all men, and their loss, therefore, is equal, for it is clear that
what they lose in death is but a fleeting instant of time. No man can
lose either the past or the future, for how can a man be deprived of
what he has not? These two things then are to be remembered: First,
that all things recur in cycles, and are the same from everlasting,
and that, therefore, it matters nothing whether a man shall
contemplate these same things for one hundred years, or for two
hundred, or for an infinite stretch of time: and, secondly, that he
who lives longest and he who dies soonest have an equal loss in
death. The present moment is all of which either is deprived, since
that is all he has. No man can be robbed of that which he has not.

15. Beyond opinion there is nothing. The objections to this saying of
Monimus the Cynic are obvious. But obvious also is the utility of what
he said, if one accept his pleasantry as far as truth will warrant it.

16. Man's soul dishonours itself, firstly and chiefly when it does all
it can to become an excrescence, and as it were an abscess on the
Universe. To fret against any particular event is to revolt against
the general law of Nature, which comprehends the order of all events
whatsoever. Again it is dishonour for the soul when it has aversion to
any man, and opposes him with intention to hurt him, as wrathful men
do. Thirdly, it affronts itself when conquered by pleasure or pain;
fourthly, when it does or says anything hypocritically, feignedly or
falsely; fifthly, when it does not direct to some proper end all its
desires and actions, but exerts them inconsiderately and without
understanding. For, even the smallest things should be referred to the
end, and the end of rational beings is to follow the order and law of
the venerable state and polity which comprehends them all.

17. The duration of man's life is but an instant; his substance is
fleeting, his senses dull; the structure of his body corruptible; the
soul but a vortex. We cannot reckon with fortune, or lay our account
with fame. In fine, the life of the body is but a river, and the life
of the soul a misty dream. Existence is a warfare, and a journey in a
strange land; and the end of fame is to be forgotten. What then avails
to guide us? One thing, and one alone--Philosophy. And this consists
in keeping the divinity within inviolate and intact; victorious over
pain and pleasure; free from temerity, free from falsehood, free from
hypocrisy; independent of what others do or fail to do; submissive to
hap and lot, which come from the same source as we; and, above all,
with equanimity awaiting death, as nothing else than a resolution of
the elements of which every being compounded. And, if in their
successive interchanges no harm befall the elements, why should one
suspect any in the change and dissolution of the whole? It is natural,
and nothing natural can be evil.

\terminus{At Carnuntum.}

\chapter[Man must consider,...]{}

1. \textsc{Man must consider,} not only that each day part of his life is
spent, and that less and less remains to him, but also that, even if
he live longer, it is very uncertain whether his intelligence will
suffice as heretofore for the understanding of his affairs, and for
grasping that knowledge which aims at comprehending things human and
divine. When dotage begins, breath, nourishment, fancy, impulse, and
so forth will not fail him. But self-command, accurate appreciation of
duty, power to scrutinize what strikes his senses, or even to decide
whether he should take his departure, all powers, indeed, which demand
a well-trained understanding, must be extinguished in him. Let him be
up and doing then, not only because death comes nearer every day, but
because understanding and intelligence often leave us before we die.

2. Observe what grace and charm appear even in the accidents that
accompany Nature's work. Thus some parts of a loaf crack and burst in
the baking; and this cracking, though in a manner contrary to the
design of the baker, looks well and invites the appetite. Figs, too,
gape when at their ripest, and in ripe olives the very approach to
rotting adds a special beauty to the fruit. The droop of ears of corn,
the bent brows of the lion, the foam at a boar's mouth, and many other
things, are far from comely in themselves, yet, since they accompany
the works of Nature, they make part of her adornment, and rejoice the
beholder. Thus, if a man be sensitive to such things, and have a more
than common penetration into the constitution of the whole, scarce
anything connected with Nature will fail to give him pleasure, as he
comes to understand it. Such a man will contemplate in the real world
the fierce jaws of wild beasts with no less delight than when
sculptors or painters set forth for him their presentments. With like
pleasure will his chaste eyes behold the maturity and grace of old age
in man or woman, and the inviting charms of youth. Many such things
will strike him, things not credible to the many, but which come to
him alone who is truly familiar with the works of Nature and near to
her own heart.

3. Hippocrates, who had healed many diseases, himself fell sick, and
died. The Chaldeans foretold the fatal hours of multitudes, and
afterwards fate carried themselves away. Alexander, Pompey, and Gaius
Caesar, who so often razed whole cities, and cut off in battle so many
myriads of horse and foot, at last departed from this life
themselves. Heraclitus, after his many speculations on the
conflagration of the world, died, swollen with water and plastered
with cow-dung. Vermin destroyed Democritus; Socrates was killed by
vermin of another sort. What of all this? You have gone aboard, made
your voyage, come to harbour. Disembark: if into another life, there
will God be also; if into nothingness, at least you will have done
with bearing pain and pleasure, and with your slavery to this vessel
so much meaner than its slave. For the soul is intelligence and deity,
the body dust and corruption.

4. Waste not what remains of life in consideration about others, when
it makes not for the common good. Be sure you are neglecting other
work if you busy yourself with what such a one is doing and why, with
what he is saying, thinking, or scheming. All such things do but
divert you from the steadfast guardianship of your own soul. It
behoves you, then, in every train of thought to shun all that is
aimless or useless, and, above all, everything officious or
malignant. Accustom yourself so, and only so, to think, that, if any
one were suddenly to ask you, ``Of what are you thinking-now?'' you
could answer frankly and at once, ``Of so and so.'' Then it will plainly
appear that you are all simplicity and kindliness, as befits a social
being who takes little thought for enjoyment or any phantom pleasure;
who spurns contentiousness, envy, or suspicion; or any passion the
harbouring of which one would blush to own. For such a man, who has
finally determined to be henceforth among the best, is, as it were, a
priest and minister of the Gods, using the spirit within him, which
preserves a man unspotted from pleasure, unwounded by any pain,
inaccessible to all insult, innocent of all evil; a champion in the
noblest of all contests--the contest for victory over every
passion. He is penetrated with justice; he welcomes with all his heart
whatever befalls, or is appointed by Providence. He troubles not
often, or ever without pressing public need, to consider what another
may say, or do, or design. Solely intent upon his own conduct, ever
mindful of his own concurrent part in the destiny of the Universe, he
orders his conduct well, persuaded that his part is good. For the lot
appointed to every man is part of the law of all things as well as a
law for him. He forgets not that all rational beings are akin, and
that the love of all mankind is part of the nature of man; also that
we must not think as all men think, but only as those who live a life
accordant with nature. As for those who live otherwise, he remembers
always how they act at home and abroad, by night and by day, and how
and with whom they are found in company. And so he cannot esteem the
praise of such, for they enjoy not their own approbation.

5. In action be neither grudging, nor selfish, nor ill-advised, nor
constrained. Let not your thought be adorned with overmuch nicety. Be
not a babbler or a busybody. Let the God within direct you as a manly
being, as an elder, a statesman, a Roman, and a ruler, standing
prepared like one who awaits the recall from life, in marching order;
requiring neither an oath nor the testimony of any man. And withal, be
cheerful, and independent of the assistance and the peace that comes
from others; for, it is a man's duty to stand upright, self-supporting,
not supported.

6. If in the life of man you find anything better than justice, truth,
sobriety, manliness; and, in sum, anything better than the
satisfaction of your soul with itself in that wherein it is given to
you to follow right reason; and with fate in that which is determined
beyond your control; if, I say, you find aught better than this, then
turn thereto with all your heart, and enjoy it as the best that is to
be found. But if nothing seems to you better than the divinity seated
within you, which has conquered all your impulses, which sifts all
your thoughts, which, as Socrates said, has detached itself from the
promptings of sense, and devoted itself to God and to the love of
mankind; if you find every other thing small and worthless compared
with this, see that you give place to no other which might turn,
divert, or distract you from holding in highest esteem the good which
is especially and properly your own. For it is not permitted to us to
substitute for that which is good in reason or in fact anything not
agreeable thereto, such as the praise of the many, power, riches, or
the pursuit of pleasure. All these things may seem admissible for a
moment; but presently they get the upper hand, and lead us astray. But
do you, I say, frankly and freely choose the best, and keep to it. The
best is what is for your advantage. If now you choose what is for your
spiritual advantage, hold it fast; if what is for your bodily
advantage, admit that it is so chosen, and keep your choice with all
modesty. Only see that you make a sure discrimination.

7. Never esteem aught of advantage which will oblige you to break your
faith, or to desert your honour; to hate, to suspect, or to execrate
any man; to play a part; or to set your mind on anything that needs to
be hidden by wall or curtain. He who to all things prefers the soul,
the divinity within him, and the sacred cult of its virtues, makes no
tragic groan or gesture. He needs neither solitude nor a crowd of
spectators; and, best of all, he will live neither seeking nor
shunning death. Whether the soul shall use its surrounding body for a
longer or shorter space is to him indifferent. Were he to depart this
moment he would go as readily as he would do any other seemly and
proper action, holding one thing only in life-long avoidance--to find
his soul in any case unbefitting an intelligent social being.

8. In the soul of the chastened and purified man you would find
nothing putrid, foul, or festering. Fate does not cut off his life
before its proper end; as one would say of an actor who left the stage
before his part was ended, or he had reached his appointed exit. There
remains nothing servile or affected, nothing too conventional or too
seclusive, nothing that fears censure or courts concealment.

9. Hold in honour the faculty which forms opinions. It depends on this
faculty alone that no opinion your soul entertains be inconsistent
with the nature and constitution of the rational being. It ensures
that we form no rash judgments, that we are kindly to men, and
obedient to the Gods.

10. Cast from you then all other things, retaining these few. Remember
also that every man lives only this present moment, which is a
fleeting instant: the rest of time is either spent or quite
unknown. Short is the time which each of us has to live, and small the
corner of the earth he has to live in.  Short is the longest
posthumous fame, and this preserved through a succession of poor
mortals, soon themselves to die; men who knew not themselves, far less
those who died long ago.

11. To these maxims add this other. Accurately define or describe
every thing that strikes your imagination, so that you may see and
distinguish what it is in naked essence, and what it is in its
entirety; that you may tell yourself the proper name of the thing
itself, and the names of the parts of which it is compounded, and into
which it will be resolved. Nothing makes mind greater than the power
to enquire into all things that present themselves in life; and, while
you examine them, to consider at the same time of what fashion is the
Universe, and what is the function in it of these things, of what
importance they are to the whole, of what to man who is a citizen of
that highest city of which all other cities are but households.
Consider what is this thing that now makes an impression on you, of
what it is composed, and how long it is destined to endure. Consider
also for what virtue it calls; whether it be gentleness, courage,
truthfulness, fidelity, simplicity, independence, or any other. Say,
therefore, of each event: ``This comes from God:'' or ``This comes from
the conjunction and intertexture of the strands of fate, or from some
chance or hazard of that kind:'' or ``This comes from one of my own
tribe, from my kinsman, from my friend. He is, indeed, ignorant of
what accords with nature; but I am not, and will therefore use him
kindly and justly, according to the natural and social law. As to
things indifferent, I strive to appraise them at their proper value.''

12. If you discharge your present duty with firm and zealous, yet
kindly, observance of the laws of reason; if you regard no by-gains,
but keep pure within you your immortal part, as if obliged to restore
it at once to him who gave it; if you hold to this with no further
desires or aversions, and be content with the natural discharge of
your present task, and with the heroic sincerity of all you say or
utter, you will live well. And herein no man can hinder you.

\newpage

13. As surgeons have ever their knives and instruments at hand for the
sudden emergencies of their art, so do you keep ready the principles
requisite for understanding things divine and human, and for doing all
things, even the least important, in the remembrance of the bond
between the two. For in neglecting this, you will scant your duty both
to Gods and men.

14. Cease your wandering, for you are not like to read again your own
memoirs, or the deeds of the ancient Greeks and Romans, or those
collections from the writings of others that you laid up for your old
age. Hasten then to your proper end. Fling away vain hopes, and, if
you have any care for yourself, fly to your own succour while yet you
may.

15. Men understand not all that is signified by the words--to steal,
to sow, to buy, to rest, to see what is to be done. For it is not the
bodily eye but another sort of sight that must discern these things.

16. We have body, soul, and intelligence. To the body belong the
senses, to the soul the passions, to the intelligence principles. To
be affected by the imagery of sense belongs to the beasts of the field
no less than to us. To be swayed by gusts of passion is common to us
with the wild beasts, with the most effeminate wretches, with Nero and
with Phalaris. Moreover, the possession of a mind to guide us to what
seems fitting is shared by us, with atheists, with traitors to their
country, and with such as shut their doors and sin. If, then, all the
rest is common as we have seen, there remains to the good man this
special excellence; to welcome with pleasure all that happens or is
ordained, not to defile the divinity enthroned in his breast, not to
perturb it with a crowd of images, but to preserve it in tranquillity,
and obey it as a God: to observe truth in all he says, and justice in
his every action. And though others may not believe that he lives thus
in simplicity, modesty, and contentment, he neither takes this
unbelief amiss from any one, nor quits the road which leads to the
true end of life, at which he ought to arrive pure, calm, ready to
take his departure, and accommodated without compulsion to his fate.

\terminus{}

\chapter[The power...]{}

1. \textsc{The power} which rules within us, when its state is accordant
with nature, so acts in every occurrence as easily to adapt itself to
all present or possible situations. It requires no set material to work
upon, but, under proper reservation, needs but the incitement to pursue,
and makes matter for its activities out of every opposition. Even so a
fire masters that which is cast upon it, and though a small flame would
have been extinguished, your great blaze quickly makes the added fuel
its own, consumes it, and grows mightier therefrom.

2. Let no action be done at random, nor otherwise than in complete
accordance with the principles involved.

3. Men seek retirement in the country, on the sea-coast, in the
mountains; and you too have frequent longings for such distractions.
Yet surely this is great folly, since you may retire into yourself at
any hour you please. Nowhere can a man find any retreat more quiet and
more full of leisure than in his own soul; especially when there is
that within it on which, if he but look, he is straightway quite at
rest. And rest I hold to be naught else but perfect order in the
soul.

Constantly, therefore, allow yourself this retirement, and so renew
yourself. Have also at hand thoughts brief and fundamental, which
readily may occur; sufficing to shut out the discordant clamour of the
world, and to send you back without fretting at the task to which you
return. For at what do you fret? At the wickedness of mankind. Recollect
the maxim that all reasoning beings are created for one another, that
to bear with them is a part of justice, and that they cannot help their
sin. Remember how many of those who lived in enmity, suspicion, and
hatred, at daggers drawn, have been stretched on their funeral pyres,
and turned to ashes. Remember and cease from your complaints.

Is it your allotted part in the world's destiny that chagrins you?
Be calm, and renew your knowledge of the alternative, that ``Either
providence directs the world, or there is nothing but unguided atoms;''
and recollect the many proofs that the Universe is as it were a state.
Do the ills of the body still have power to touch you? Reflect that
the mind, once withdrawn within itself, once grown conscious of its
own power, has no concern with the motions, rough or smooth, of the
breathing body. Remember, too, all that you have heard and assented
to concerning pain and pleasure. Are you distracted by the poor thing
called fame? Think how swiftly all things are forgotten. Behold the
chaos of eternity which besets us on either side. Think how empty is
the noisy echo of acclamation; how fickle and how scant of judgment are
they who would seem to praise us, and how narrow the bounds within which
their praise is confined. All the earth is but a point in the Universe;
how small a corner of that little is inhabited, and even there how few
are they and of how little worth who are to praise us! Remember then
that there ever remains for you retirement into the little field within.

And, above all, be neither distraught nor overstrained. Hold fast your
freedom: consider all things as a man of courage, as a human being, as
a citizen, as a mortal. Readiest among the principles to which you look
let there be these two: Firstly, things external do not touch the soul,
but remain powerless without; and all trouble comes from what we think
of them within. Secondly, all things visible change in a moment, and are
gone for ever. Recollect all the changes of which you have yourself been
a witness. The world is a succession of changes: life is but thought.

4. If mind be common to us all, the reason in virtue of which we are
rational is also common; so too is the power which bids us do or not
do. Therefore we have all a common law; and if so, we are
fellow-citizens and members of some common polity. The Universe, then,
must in a manner be a state, for of what other common polity can all
mankind be said to be members? Wherefore it is from this common state
that we derive our intellectual power, our reason, and our law; or
whence do we derive them? For that which is earthy in me is derived
from earth, my moisture from some other element, my breath and what is
warm or fiery from their proper sources. And therefore, as nothing can
arise from nothing or return thereto, my intellectual part has also a
source.

5. Death, like birth, is a mystery of nature; the one a compounding of
elements, the other a resolution into the same. In neither is there
anything shameful or against the nature of the rational animal, or
contrary to the law of its constitution.

6. It is fate that such actions should come from such men. He who
would have it otherwise would have figs without juice. This, too, you
should remember: that in a very short time both you and he must die;
and a little after not even the name of either shall remain.

7. Suppress the thought; and the cry ``I am hurt!'' is gone. Suppress ``I
am hurt!'' and you suppress the injury.

8. What makes not a man worse than he was, makes not his life worse,
nor hurts him without or within.

9. The law of utility must act so.

10. All that happens, happens right: you will find it so if you
observe narrowly. I mean not only according to a natural order, but
according to our idea of justice, and, as it were, by the action of
one who distributes according to merit. Go on then observing this as
you have begun, and whatever you do, let your aim be goodness,
goodness as it is rightly understood. Hold to this in every action.

11. Think not as your insulter judges or wishes you to judge: but see
things as they truly are.

12. For two things be ever ready: First, to do that only which reason,
the sovereign and legislative faculty, suggests for the good of
mankind: Secondly, to change your course on meeting any one who can
correct and alter your opinion. But let the change be made because you
really believe it to be in the interest of justice or the public good,
or such like, and not with any view to pleasure or glory for yourself.

13. Have you reason? I have. Why then do you not use it? When it
performs its proper office what more do you require?

14. You exist as part of a whole. You will disappear again in that
which produced you; or rather you will change and be resumed again
into the productive intelligence.

15. Many grains of frankincense are laid on the same altar. One falls
soon, another later. It makes no difference.

16. Within ten days, if you return to the observance of moral
principles and to the cult of reason, you will appear a God to them
who now esteem you a wild beast or an ape.

17. Order not your life as though you had ten thousand years to
live. Fate hangs over you. While you live, while yet you may, be good.

18. How much he gains in leisure who looks not to what his neighbours
say, or do, or intend; but considers only how his own actions may be
just and holy, looking not, as Agathon says, to the moral example of
others, but running a straight course and never turning therefrom.

19. He who is careful and troubled about the fame which is to live
after him considers not that each one of those who remember him must
very soon die himself, and thereafter also the succeeding generation,
until every memory of him, handed on by excited and ephemeral
admirers, dies utterly away. Grant that your memory were immortal, and
those immortal who retain it; yet what is that to you? I ask not, what
is that to the dead? But to the living what is the profit in praise,
except it be in some convenience that it brings? And you now abandon
what nature has put in your power in order to set your hopes upon the
report of others.

20. Whatever is beautiful at all is beautiful in itself. Its beauty
ends there, and praise has no part in it. Nothing is the better or the
worse for being praised; and this holds also of what is beautiful in
the common estimation: of material forms and works of art. Thus true
beauty needs nothing beyond itself, any more than law, or truth, or
kindness, or honour. For none of these gets a single grace from praise
or one blot from censure. Does the emerald lose its virtue if one
praise it not? Can one by scanting praise depreciate gold, ivory, or
purple, a lyre or a dagger, a flower or a shrub?

21. If our souls survive us, how, you ask, has the air contained them
from eternity? How, I answer, does the earth contain so many bodies
buried during so long a time? Just as corpses, after remaining for a
while in the earth, change, and are dissipated to make room for
others; so also the souls, liberated into air, remain for a little,
and then are changed, diffused, rekindled, and resumed into the
universal productive spirit; and so give way to others who come to
take their places. This may serve for an answer, on the supposition
that the soul survives the body. But we have not merely to consider
the number of bodies thus buried in the earth. There are also all the
living creatures eaten day by day by ourselves and other animals. How
great a multitude of them is thus consumed, and as it were buried in
the bodies of those who feed upon them. Yet there is ever space to
contain them, owing to the changes into blood, air, and fire. What,
then, is the key to this enquiry? Discrimination of matter and cause.

22. Swerve not from your path. In every impulse render justice its
due, and in all thinking be sure that you understand.

23. I am in tune with all that is of thy harmony, O Nature. For me
nothing is too early and nothing is too late that comes in thy good
time. All is fruit to me, O Nature, that thy seasons bring. From thee
are all things, thou comprehendest all, and all returns to thee. The
poet says, ``O dear City of Cecrops!'' Shall I not say, ``Dear City of
God''!

24. ``Do few things,'' says the philosopher, ``if you would have quiet''.
This is perhaps a better saying, ``Do what is necessary, do what the
reason of the being that is social in its nature directs, and do it in
the spirit of that direction.'' By this you will attain the calm that
comes from virtuous action, and that calm also which comes from having
few things to do. Most things you say and do are not necessary. Have
done with them, and you will be more at leisure and less perturbed. On
every occasion, then, ask yourself the question, Is this thing not
unnecessary? And put away not only unnecessary deeds but unnecessary
thoughts, for by so doing you will avoid all superfluous actions.

25. Make trial how the life of a good man succeeds with you, the life
of one who is content with the lot appointed him by Providence, and
satisfied with the justice of his own actions and the benevolence of
his disposition.

26. You have seen the other state, make trial also of this. Avoid
perplexity; seek simplicity. Has a man sinned? He bears his own
sin. Has aught befallen you? It is well; for all that befalls you is
an ordained part in the weaving of the destiny of all things from the
beginning. In sum, life is short. Make the best of the present in
reason and in justice. Be sober in your relaxation.

27. The Universe is either an ordered whole or a confusion. But,
although a mixture of phenomena, it is certainly an ordered whole. Or,
do you think that there can be order in you and confusion in the
Universe, and that too when all things, though diffused and separated,
are all in sympathy, one with another?

28. Consider the deformity of these characters: the black or
malicious, the effeminate, the savage, the beastly, the childish, the
brutish, the stupid, the false, the ribald, the knavish, the
tyrannical.

29. He is a foreigner, and not a citizen of the world, who knows not
what the world contains; and he, too, who knows not what happens in
it. He is a deserter who flies from the reason that rules this
polity. He is blind, whose intellectual eye is closed. He is a beggar,
who needs the gifts of others, and has not from himself all that is
necessary for life. He is an excrescence on the scheme of things, who
withdraws and separates himself from the reasoned constitution of the
nature in which he shares, by discontent with what befalls. That same
nature which produces this event produced thee. He is the seditious
citizen who separates his particular soul from the one soul of all
reasonable beings.

30. One acts the philosopher without a coat, another without books, a
third half-naked. Says one, ``I have not bread, and yet I hold to
reason.'' Says another, ``I have not even the spiritual food of
instruction, and yet I hold to it''.

31. Love the art which you have learned, humble though it be, and in
it find your recreation. And spend the remainder of your life as one
who with all his heart commits his concerns to the Gods, and neither
acts the tyrant nor the slave to any of mankind.

32. Recall, for example, the age of Vespasian. It is as the spectacle
of our own time. You will see men marrying, bringing up children, sick
and dying, warring and feasting, trading and farming. You will see men
flattering, obstinate in their own will, suspecting, plotting, wishing
for the death of others, repining at fortune, courting mistresses,
hoarding treasure, pursuing consulships and kingdoms. Yet all that
life is spent and gone. Come down to Trajan's days. Again all is the
same; and again, that life, too, is dead. Consider, likewise, the
records of other times and nations, and see how, after their fit of
eagerness, all quickly fell, and were resolved into the elements. But
most of all, remember those whom you yourself have known, men who were
distracted about vain things, men who neglected the course which
suited their own nature, neither holding fast to it nor finding their
contentment there. And, herein, forget not that care is to be bestowed
on any enterprise only in proportion to its proper worth. For if you
keep this in mind you will not be disheartened from over concern with
things of less account.

33. The familiar phrases of old days are now strange and obsolete;
and, likewise, the names of such as were once much celebrated now
sound strangely in our ears. Camillus, Caeso, Volesus, Leonnatus;
after them Scipio and Cato; lastly, Augustus, Hadrian, and
Antonine--all are forgotten. All things hasten to an end, shall
speedily seem old fables, and then be buried in oblivion. This I say
of those who have shone with the brightness of their fame. The rest of
men, as soon as they expire, are unknown and forgotten. What, then, is
it to be remembered for ever? A wholly empty thing. For what should we
be zealous? For this alone, that our souls be just, our actions
unselfish, our speech ever sincere, and our disposition such as may
cheerfully embrace whatever happens, seeing it to be inevitable,
familiar, and sprung from the same source and origin as we ourselves.

34. Willingly resign yourself to Clotho, permitting her to spin her
thread of what yarn she may.

35. All things are for a day, both what remembers and what is
remembered.

36. Observe continually that all things exist in change; and keep this
thought ever with you, that Nature loves nothing more than changing
what things now are, and making others like them. For what now is, is
in a manner the seed of what shall be. Therefore, conceive not that
that alone is seed which is cast into the earth or the womb, for that
is the thought of ignorance.

37. You are presently to die, and yet you have not attained to
simplicity or calm, or to disbelief that you can be hurt by things
external. You have not learned to be kindly to all men, or to count
just dealing the whole of wisdom.

38. Scan closely that which governs men; see what are their cares, and
what they pursue or shun.

39. That which is evil for you exists not in the soul of another; nor
in any change or alteration of the body which surrounds you. Where,
then, is it? It lies in that part of you by which you apprehend what
evil is. Stay the apprehension, and all is well. And though the poor
body to which it is so closely bound be cut and burned, though it
suppurate or mortify, yet let the apprehension remain inactive: that
is, let it judge nothing either bad or good which can happen equally
to the bad man and to the good. For that which befalls equally him who
lives in accord, and him who lives in discord with Nature, can neither
be natural nor unnatural.

40. Ever consider this Universe as one living being, with one material
substance and one spirit. Observe how all things are referred to the
one intelligence of this being; how all things act on one impulse; how
all things are concurrent causes of all others; and how all things are
connected and intertwined.

41. ``Thou art a poor soul, saddled with a corpse,'' said Epictetus.

42. There is no evil for things which subsist in change; and there can
be no good for things which subsist without it.

43. Time is a river, a violent torrent of things coming into
being. Each one, as soon as it has appeared, is swept away: it is
succeeded by another which is swept away in its turn.

44. All that happens is as natural and familiar as a rose in spring,
or fruit in summer. Such are disease and death, calumny and treachery,
and all else which gives fools joy or sorrow.

45. Consequents follow antecedents by virtue of a special and
necessary connexion. This relation is not that which exists in a mere
enumeration of independent things, and depends merely on some
arbitrary convention. It is a rational relationship. And just as
things now existing are ranged harmoniously together, so those which
come into existence display no bare succession, but a wonderful
harmony with what preceded.

46. Remember always the sayings of Heraclitus: that the death of earth
is to become water, the death of water to become air, and the death of
air to become fire; and so conversely. Remember in what a case he is
who forgets whither the way leads: that men are frequently at variance
with their close and constant companion, the reason which rules all:
that men count strange that which they meet every day: that we should
neither act nor speak as though in slumber, although even in slumber
we seem to act and speak; nor yet like children learning from their
parents, with a mere acceptance of everything just as we are told it.

47. If some God were to inform you that you must die tomorrow, or the
next day at farthest, you would take little concern whether it was to
be tomorrow or the next day; that is if you were not the most
miserable of cowards. For how small is the difference? Wherefore,
account it of no great moment whether you die after many years or
tomorrow.

48. Constantly consider how many physicians are dead and gone, who
frequently knitted their brows over their patients; how many
astrologers, who foretold the deaths of others with great ostentation
of their art; how many philosophers, who wrote endlessly on death and
immortality; how many warriors, who slew their thousands; and how many
tyrants, who used their power of life and death with cruel wantonness,
as though they had been immortal. How many whole cities, if I may so
speak, are dead: Helice and Pompeii and Herculaneum, and others past
counting. Tell over next all those you have known, one after the
other: think how one buried his fellow, then lay dead himself, to be
buried by a third. And all this within a little time. In sum, look
upon human things, and behold how short-lived and how vile they are;
mucus yesterday, tomorrow ashes or pickled carrion. Spend, then, the
fleeting remnant of your time in a spirit that accords with Nature,
and depart contentedly. So the olive falls when it is grown ripe,
blessing the ground from whence it sprung, and thankful to the tree
that bore it.

49. Be like a promontory against which the waves are always
breaking. It stands fast, and stills the waters that rage around
it. ``Wretched am I,'' says one, ``that this has befallen me.'' ``Nay,'' say
you, ``happy am I who, though this has befallen me, can still remain
without sorrow, neither broken by the present nor dreading the
future.'' The like might have befallen any one; but every one would not
have endured it unpained. Why, then, should we dwell more on the
misfortune of the incident than on the felicity of such strength of
mind? Can you call that a misfortune for a man which is not a
miscarriage of his nature?  And can you call anything a miscarriage of
his nature which is not contrary to its purpose? You have learned its
purpose, have you not?  Then does this accident debar you from
justice, magnanimity, prudence, wisdom, caution, truth, honour,
freedom, and all else in the possession of which man's nature finds
its full estate? Remember, therefore, for the future, upon all
occasions of sorrow, to use the maxim: this thing is not misfortune,
but to bear it bravely is good fortune.

50. It is a vulgar meditation, and yet very effectual for enabling us
to despise death, to consider the fate of those who have been most
earnestly tenacious of life, and enjoyed it longest. Wherein is their
gain greater than that of those who died before their time? They are
all lying dead somewhere or other. Cadicianus, Fabius, Julian,
Lepidus, and their fellows, saw the corpses of multitudes carried to
the grave, and then themselves were carried thither. In sum, how small
was the difference of time, spent painfully amid what troubles, among
what worthless men, and in how mean a carcase! Think it not a thing of
value. Rather look back into the eternity that gapes behind, and
forward into the other abyss of immensity. Compared with such
infinity, small is the difference between a life of three days and one
of three ages like Nestor's.

51. Run ever the short way. The short way is the way according to
Nature. Therefore speak and act according to the soundest rule; for
this resolution will free you from much toil and warring, and from all
artful management and ostentation.

\terminus{}

\chapter[In the morning,...]{}

1. \textsc{In the morning,} when you find yourself unwilling to rise,
have this thought at hand: I arise to the proper business of man, and
shall I repine at setting about that work for which I was born and
brought into the world? Am I equipped for nothing but to lie among the
bed-clothes and keep warm? ``But,'' you say, ``it is more pleasant so''.
Is pleasure, then, the object of your being, and not action, and the
exercise of your powers? Do you not see the smallest plants, the little
sparrows, the ants, the spiders, the bees, all doing their part, and
working for order in the Universe, as far as in them lies? And will you
refuse the part in this design which is laid on man? Will you not pursue
the course which accords with your own nature? You say, ``I must have
rest.'' Assuredly; but nature appoints a measure for rest, just as for
eating and drinking. In rest you go beyond these limits, and beyond what
is enough; but in action you do not fill the measure, and remain well
within your powers. You do not love yourself; if you did, you would love
your nature and its purpose. Others, who love the art that they have
made their own, exhaust themselves with labouring at it unwashed and
unfed. But you honour your own nature less than the carver honours his
carving, less than the dancer honours his dancing, the miser his gold,
or the vain man his empty fame. These men, when desire takes them, count
food and sleep well lost if they can better realize the object of their
longings; and shall the pursuit of the common good seem less precious in
your eyes and worthy of a lesser zeal?

2. How easy it is to thrust away and blot out each impression that is
disturbing and unfit; and forthwith to enjoy perfect tranquillity.

3. Judge no speech or action unworthy of you which is consistent with
nature. Be not dissuaded by any consequent criticism or censure from
others; but, if the speech or action be honourable, judge yourself
worthy to say or do it. Those who criticize you have their own
conscience and their own motives. These you are not to regard, but
follow a straight course, guided by your own nature and the nature of
the Universe, both of which point the same way.

4. I walk the way which is Nature's, until at last I shall fall and be
at rest; breathing out my breath into the air wherefrom I daily drew
it, falling on that earth whence my father drew his seed, my mother
her blood, and my nurse the milk which nourished me; on that earth
which has given me my daily food and drink through all these years,
which sustains my footsteps, and bears with me--her manifold abuser.

5. Men cannot admire you for your shrewdness. Be it so. But there is
many another quality of which you cannot say, ``It is not in me''.
Display these; they are wholly in your power. Be sincere, be
dignified, be painstaking; scorn pleasure, repine not at fate, need
little; be kind and frank; love not exaggeration and vain talk; strive
after greatness. Do you not see how many virtues you might show, of
which you are yet content to fall short, though you have not the
excuse that they are absent, or that you are unfit for them? Are you
driven by some want in your equipment to be querulous, to be miserly,
to be a flatterer, to reproach your body with your own faults, to
cringe to others, to be vainglorious, to have all this restlessness in
your soul? No, by the Gods, you might have escaped these vices long
ago. All your fault, then, is that you are somewhat slow and dull of
comprehension. This you should strive to correct by exercise; neither
neglecting your dulness nor taking a mean pleasure in it.

6. Some men, when they have done you a favour, are very ready to
reckon up the obligation they have conferred. Others, again, are not
so forward in their claims, but yet in their minds consider you their
debtor, and well know the value of what they have done. A third sort
seem to be unconscious of their service. They are like the vine, which
produces its clusters and is satisfied when it has yielded its proper
fruit. The horse when he has run his course, the hound when he has
followed the track, the bee when it has made its honey, and the man
when he has done good to others, make no noisy boast of it, but set
out to do the same once more, as the vine in its season produces its
new clusters again. ``Should we, then, be among those who in a manner
know not what they do?'' Assuredly. ``But this very thing implies
intelligence; for it is a property of the unselfish man to perceive
that he is acting unselfishly, and, surely, to wish his fellow also to
perceive it.'' True, but if you misapprehend my saying, you will enter
the ranks of those of whom I spoke before. They, too, are led astray
by specious reasonings. But if you have the will to understand what my
principle truly means, fear not that in following it you will neglect
the duty of unselfishness.

7. This is a prayer of the Athenians: ``Rain, rain, dear Zeus, on the
plains and ploughlands of the Athenians.'' Man should either not pray
at all, or pray after this frank and simple fashion.

8. Just as one says that Aesculapius has prescribed a course of riding
for some one, or the cold bath, or walking bare-footed; so it may be
said that the guiding Mind prescribes for a man, disease, or
mutilation, or losses, or the like. ``Prescribed,'' in the first case,
means that such treatment was enjoined on the patient as might
coincide with the needs of his health: in the second case it means
that each man's fortune is appointed to coincide with the purposes of
fate. Now, the very word ``coincidence'' implies something like that
correspondence of squared stones in a wall or pyramid, which workmen
speak of when they fit them together in some structure. All things are
united in one bond of harmony; and just as all existing bodies go to
make the visible world what it is, so destiny, as the general cause,
is compounded of all particular causes. The most unphilosophical grasp
my meaning, for they say, ``Fate gave this to so-and-so: this was
appointed or prescribed for him.'' Let us, then, receive the decrees of
Fate as we receive the prescriptions of Aesculapius. He prescribes
many things for us, and some of them are harsh medicines. Yet we obey
him gladly in the hope of health. Conceive therefore that, for Nature,
the doing of her work and the fulfilling of her purposes are, as it
were, her health; and welcome all that happens, even should it seem
hard fortune, because it tends to the health of the Universe, and to
the prosperity and felicity of Zeus. He would not have brought this or
that on any man did it not contribute to the good of the whole, nor
does any part of Nature's system bring aught to pass which suits not
with her government. For two reasons, then, you should content
yourself with what befalls you. The first is, that it was created and
ordained for you, and was in a manner related to you from the
beginning, in the weaving of all destinies from the great first
causes. The second is, that even what happens severally to each man
contributes to the well-being and prosperity of the Mind which governs
all things, and, indeed, even to its continued existence. For the
whole is maimed if you break in the slightest degree this continuous
connexion, whether of parts or causes. And this you are doing your
best to break and to destroy whenever you repine at fate.

9. Fret not, neither despond nor be disheartened, if it be not always
possible for you to act according to your principles of perfection. If
you are beaten off, return again to the effort, and content yourself
that your conduct is generally such as becomes a man. Love the good to
which you return; and come back to Philosophy, not as one who comes to
a master, but as one whose eyes ache recurs to sponge and egg, as
another has recourse to plasters, or a third to fomentation. And thus
you will make no empty show of obeying reason; but find that it gives
you rest. Remember that Philosophy demands no more than what your
nature requires. But you are wont to desire other things which accord
not with your nature. ``For what,'' you say, ``can be more delightful
than such things?'' Is not this the very snare which Pleasure sets for
us? Yet consider if magnanimity, frankness, simplicity, kindness, and
piety be not even greater delights. And what is sweeter than wisdom
itself, when you are conscious of security and felicity in your powers
of apprehension and reason?

10. The natures of things are so covered up from us, that to many
philosophers, and these no mean ones, all things seem incomprehensible.
The Stoics themselves own that it is difficult to comprehend anything
with certainty. All our assent is inconsistent, for where is the
consistent man? Consider, too, the objects of our knowledge: how
transitory are they, and how mean! How often they are in the
possession of the debauchee, of the harlot, of the robber!  Review
again the morals of your contemporaries: it is scarcely possible to
tolerate the best-mannered among them; not to say that a man can
scarcely tolerate himself. Amid such darkness and filth, in this
perpetual flux of substance, of time, of motion, and of things moved,
I can perceive nothing worthy of esteem or of desire. On the contrary,
we should comfort ourselves as we await our natural dissolution, and
not be vexed at the delay, but find rest in these thoughts: first,
that nothing can befall us which is not in accord with the nature of
all things; second, that it is always in our power not to do anything
against the divine spirit within us: to this no force can compel us.

11. To what end am I using my soul? Let me examine myself as to this
on all occasions, and consider what is passing now in that part of me
which men call the ruler of the rest. Let me think, too, whose is the
soul that I have. Is it a child's? Is it a youth's, a timorous
woman's, or a tyrant's; the soul of a tame beast or of a savage one?

12. Of what value the things are which the many account good you may
judge from this: If a man has conceived certain things, such as
prudence, temperance, justice, or courage, to be good in the real
sense, he cannot, while he is of this mind, readily listen to the
traditional gibe about a superabundance of good things. It will not
fit the case. But when he has in mind things which seem good in the
eyes of the multitude, he is perfectly willing to hear and accept as
quite appropriate the raillery of the comic poet. Thus even the
ordinary mind perceives the difference. For if this were not so, we
would not in the first case repudiate the jest as offensive, nor would
we salute it as a happy witticism when applied to wealth or to the
opulence which produces luxury and ostentation. Proceed then, and put
the question whether these things are to be valued and esteemed good
of which we have such an opinion that we may aptly say of their
possessor: ``He has so many possessions about him that he has no place
wherein to ease himself''.

13. I consist of a formal and a material element. Neither of these two
shall die and fade into nothingness, since neither came into being out
of nothing. Every part of me, then, will be transformed and ranged
again in some part of the Universe. That part of the Universe will
itself be transmuted into another part, and so on for all time
coming. By some such change as this I came into being, likewise my
progenitors, and so back from all time past. There is no objection to
this theory, even though the world be governed by determined cycles of
revolution.

14. Reason, and the art of thinking, are powers which are complete in
themselves, and in their special processes. They start from their own
internal principle, and proceed to their appointed end. Such mental
acts are called right, to indicate that the course of thought is right
or straight.

15. Nothing should be said to be part of a man which is not part of
his human nature. Things that are not part of his essence cannot be
required of him, and have no part in the promise or the fulfilment of
his nature. Therefore, in such things lies neither the end of man nor
the good which crowns that end. Moreover, if anything were really part
of a man, it would not be proper for him to despise it or revolt
against it, nor would he be praiseworthy who made himself independent
thereof. If non-essential things were indeed good, he could be no good
man who stinted himself in the use of them; but, as we see, the more a
man goes without them, and the more he endures the want of them, the
better a man he is.

16. The character of your most frequent impressions will be the
character of your mind. The soul takes colour from its impressions,
therefore steep it in such thoughts as these:--Wherever a man can
live, he can live well. A man can live in a court, therefore he can
live well there. Again everything works towards that for which it was
created, and that to which anything works is its end; and in the end
of everything is to be found the advantage and the good of it. Now,
for reasoning beings, Society is the highest good, for it has long
since been proved that we were brought into the world to be
social. Nay, was it not manifest that the inferior kinds were formed
for the superior, and the superior for each other? Now, the animate is
superior to the inanimate, and beings that reason to those that only
live.

\newpage

17. To pursue impossibilities is madness; and it is impossible that
the wicked should not act in some such way as this.

18. Nothing can befall any man which he is not fitted by nature to
bear. The like events befall others, and either through ignorance that
the event has happened, or from ostentation of magnanimity, they stand
firm and unhurt by them. Strange then that ignorance or ostentation
should have more strength than wisdom!

19. Material things cannot touch the soul at all, nor have any access
to it: neither can they bend or move it. The soul is bent or moved by
itself alone, and remodels all things that present themselves from
without in accordance with whatever judgment it adopts within.

20. In one respect man is nearest and dearest to me; in so far, that
is, as I must do good to him and bear with him. But in so far as some
men obstruct me in my natural activities, man enters the class of
things indifferent to me, no less than the sun, the wind, or the wild
beast. By these indeed some special action may be impeded, but no
interference with my purpose or with my inward disposition can come
from them, thanks to my exceptive and modifying powers. For the mind
can convert and change everything that impedes its activity into
matter for its action; hindrance in its work becomes its real help,
and every obstruction makes for its progress.

21. Reverence that which is most excellent in the Universe, and the
most excellent is that which employs all things and rules
all. Likewise reverence that which is most excellent in yourself. It
is of the same nature as the former, for it is that which employs all
else that is in you, and that by which your whole life is ordered.

22. That which harms not the city cannot harm the citizen. Apply this
rule whenever you have the idea that you are hurt. If the state be not
hurt by this, neither am I harmed, and if the state be hurt we should
not be wrathful with him who hurt it. Consider where lay his
oversight.

23. Consider frequently how swiftly things that exist or are coming
into existence are swept by and carried away. Their substance is as a
river perpetually flowing; their actions are in continual change, and
their causes subject to ten thousand alterations. Scarcely anything is
stable, and the vast eternities of past and future in which all things
are swallowed up are close upon us on both hands. Is he not then a
fool who is puffed up with success in the things of this world, or is
distracted, or worried, as if he were in a time of trouble likely to
endure for long.

24. Keep in mind the universe of being in which your part is exceeding
small, the universe of time of which a brief and fleeting moment is
assigned to you; the destiny of things, and how infinitesimal your
share therein.

25. Does another wrong me? Let him look to that. His character and his
actions are his own. So much is in my present possession as is
dispensed to me by the nature of things, and I act as my own nature
now bids me.

26. Let the leading and ruling part of your soul stand unmoved by the
stirrings of the flesh, whether gentle or rude. Let it not commingle
with them, but keep itself apart, and confine these passions to their
proper bodily parts; and if they rise into the soul by any sympathy
with the body to which it is united, then we must not attempt to
resist the sensation, seeing that it is of our nature; but let not the
soul, for its part, add thereto the conception that the sensation is
good or bad.

27. Live with the Gods. And he lives with the Gods who continually
displays to them his soul, living in satisfaction with its lot, and
doing the will of the inward spirit, a portion of his own divinity
which Zeus has given to every man for a ruler and a guide. This is the
intelligence, the reason that abides in us all.

28. Are you angry with one whose armpits smell or whose breath is
foul?  What is the use? His mouth or his arm-pits are so, and the
consequence must follow. But, you say, man is a reasonable being, and
could by attention discern in what he offends. Very well, you too have
reason. Use your reason to move his; instruct, admonish him. If he
listens, you will cure him, and there will be no reason for anger. You
are neither actor nor harlot.

29. As you intend to live at your going, so you can live here. But, if
men do not permit you, then depart from life, yet so as if no
misfortune had befallen you. If my house be smoky, I go out, and where
is the great matter? So long as no such trouble drives me out, I
remain at my will, and no one will prevent me from acting as I
will. And my will is the will of a reasonable and social being.

30. The intelligence of the Universe is social. It has therefore made
the inferior orders for the sake of the superior; and has suited the
superior beings for one another. You see how it hath subordinated, and
co-ordinated, and distributed to each according to its merit, and
engaged the nobler beings to a mutual agreement and unanimity.

31. How have you behaved towards the Gods, towards your parents, your
brothers, your wife, your children, your teachers, those who reared
you, your friends, your intimates, your slaves? Can it be said that
you have ever acted towards all of them in the spirit of the line:--

\begin{quote}
He wrought no harshness, spoke no unkind word?
\end{quote}

Recollect all you have passed through, all that you have had strength
to bear. Your life is now a tale that is told, and your service is all
discharged. Recall the fair sights you have seen, the pleasures and
the pains you have despised, the so-called glory that you have
foregone, the unkindly men to whom you have shown kindness.

32. How is it that unskilled and ignorant souls disturb the skilful
and intelligent? What, I ask, is the skilful and intelligent soul? It
is that which knows the beginning and the end, and the reason which
pervades all being, and by determined cycles rules the Universe for
all time.

33. In a little space you will be only ashes and dry bones and a name,
perhaps not even that. A name is but so much empty sound and echo, and
the things which are so much prized in life are empty, mean, and
rotten. We are as puppies that snap at one another, as children that
quarrel, laugh, and presently weep again. But integrity, modesty,
justice, and truth,

\begin{quote}
Up from the wide-wayed earth have soared to heaven.
\end{quote}

What then should detain you here? Things sensible are ever changing
and unstable. The senses are dull and easily deceived. The poor soul
itself is a mere exhalation from blood. Fame in such a world is a
thing of naught. What then? You await calmly extinction or
transformation, whichever it may be. And till the fulness of the time
be come what is to suffice you? What else than a life spent in fearing
and praising the Gods, and in the practice of benevolence, toleration
and forbearance towards men? And whatsoever lies beyond the bounds of
flesh and breath, remember that it is neither yours nor in your power.

34. A prosperous life may be yours if only you can take the right
path, and keep to it in all you think or do. Two advantages are common
to Gods, to men, and to every rational soul. In the first place,
nothing external to themselves has power to hinder them. In the
second, their happiness lies in having mind and conduct disposed to
justice, and in the power to make that the end of all desire.

35. If the fault be not my sin, nor a consequence of it, if there be
no damage to the common good, why am I perturbed about it? Wherein is
the harm to the common good?

36. Be not incautiously carried away by sentiment, but aid him that
needs it according to your power and his desert. If his need be of the
things which are indifferent, think not that he is harmed thereby, for
so to think is an evil habit. But as, in the Comedy, the old man begs
to have his fosterchild's top for a keepsake, though he knows well
that it is a top and nothing more, so should you act also in the
affairs of life.

You mount the rostra and cry aloud, ``O man, have you forgotten what is
the real value of what you seek?'' ``No, but the many are keen in their
pursuit of it.'' ``Are you then to be a fool because they are''?

\newpage

In whatever case I had been left I could have made my fortune: for
what is it to make a fortune but to confer good things upon one's
self; and true good things are a worthy frame of mind, worthy
impulses, worthy actions.

\terminus{}

\chapter[The substance of the Universe...]{}

1. \textsc{The substance of the Universe} is docile and pliable. The
mind which governs it has in itself no source of evil-doing. It has no
malice: it does no ill, and nothing is hurt by it. By its guidance all
things come to be, and fulfil their being.

2. Act the part which is worthy of you, regarding not whether you be
stiff with cold or comfortably warm; whether you be drowsy or
refreshed with sleep; whether you be in good report or bad; whether
you be dying or upon some other business. For death also is one piece
of the business of life, and, here as elsewhere, it is enough to do
well what comes to our hand.

3. Look within. Let not the proper quality or value of anything escape
you.

4. All that exists will very speedily change; by rarefaction, if all
substance be one; otherwise by dispersion.

5. The guiding mind knows what its own condition is, how, and upon
what matter its work is done.

6. The best revenge is not to copy him that wronged you.

\newpage

7. Find your sole delight and recreation in proceeding from one
unselfish action to another, with God ever in mind.

8. The ruling part of you is that which rouses and steers itself,
making itself what it wishes to be, and making all that happens take
such appearance as it will.

9. All things are accomplished according to the will of universal
nature. There is no other nature to influence them which either
comprehends the former from without, or is contained within it, or
exists externally, and independent of it.

10. The Universe is either a confusion ravelled and unravelled again,
or else a unity compact of order and forethought. If it be the former,
why should I wish to linger amid this aimless chaos and confusion, or
have any further care than ``how to become earth again''? Nay, why am I
disturbed at all? Dissolution will overtake me, do what I please. But,
if the latter be the case, I adore the Ruler of all things, I stand
firm, and put my trust in him.

11. Whenever your situation forces trouble upon you, return quickly to
yourself, and interrupt the rhythm of life no longer than you are
compelled. Your grasp of the harmony will grow surer by continual
recurrence to it.

12. Had you at one time both a step-mother and a mother, you would
respect the former, yet you would be more constantly in your mother's
company. Your court and your philosophy are step-mother and mother to
you. Return then frequently to your true mother, and recreate yourself
with her. Her consolation can make the court seem bearable to you, and
you to it.

13. Keep these thoughts for meats and eatables: This that is before me
is the dead carcase of a fish, a fowl, a hog. This Falernian is but a
little grape juice. Think of your purple robes as sheep's wool stained
in the blood of a shell-fish. Such conceptions, which touch reality so
near, and set forth the sum and substance of these objects, are
powerful indeed to display to us their despicable value. In this
spirit we should act throughout life; and when things of great
apparent worth present themselves, we should strip them naked, view
their meanness, and cast aside the glowing description which makes
them seem so glorious. Vanity is a great sophist, and most imposes on
us when we believe ourselves to be busy about the noblest
ends. Remember the saying of Crates about Xenocrates himself.

14. Most objects of vulgar admiration may be referred to certain
general classes. There are, first, those which hold together by
cohesion or by some organic unity, such as stone, timber, figs, vines
or olives. The things which men, a shade more reasonable, admire are
referred to the class which possesses animal life such as is seen in
flocks and herds. When man's taste is still more cultured his
admiration turns to things which can show a rational intelligence. But
he admires this intelligence not as a universal principle, but only so
far as he finds it expressed in art or industry, or, indeed, sometimes
merely so far as it is exhibited by his retinue of artist slaves. But
he who values rational intelligence as a universal thing, and as a
social force, will care nothing for these other objects of
admiration. He will, above all things, strive to preserve his own mind
in all its rational and social instincts and activities; and to this
end he will co-operate with any of his kind.

\newpage

15. Some things hasten into being. Some hasten to be no more. Even as
a thing is born some part of it is already dead. Flux and change are
constantly renewing the world, just as the unbroken flow of time ever
presents to us some new portion of eternity. In this vast river, on
whose bosom there is no tarrying, what is there among the things that
sweep by us that is worth the prizing? It is as if a man grew fond of
one among a passing flight of sparrows, when already it had vanished
from his sight. Our life itself is much like a vapour of the blood or
a drawing in of air. Our momentary actions of inhalation and
exhalation are one in kind with that whole power of breathing which,
yesterday or the day before, we received at birth, and which we must
restore again to the source from whence we drew it.

16. It is a small privilege to transpire like plants, or even to
breathe as cattle or wild beasts do. To feel the impressions of sense,
to be swayed like puppets by passion, to herd together and to live by
bread; all this is no great thing. There is nothing here superior to
our power of discharging our superfluous food. What, then, is of
value? To be received with clapping of hands? No. Neither, therefore,
is the applause of tongues more valuable, for the praises of the
multitude are naught but the idle clapping of tongues. Dismiss the
vanity called fame, and what remains to be prized? This, I think: in
all things to act, or to restrain yourself from action, as best suits
the particular structure of your nature. This is the end of all arts
and studies, for every art aims at making what it produces well
adapted to the work for which it was designed. The gardener, the
vine-dresser, the horse-breaker, the dog-trainer all try for this; and
what else is the aim of all education and teaching? Here, then, is
what you may truly value: this well won, you will seek for nothing
more. Will you, then, cease valuing the multitude of other things? If
you do not, you will never attain to freedom, self-sufficiency, or
tranquillity. You cannot escape envying, suspecting, and striving
against those who have the power to deprive you of your cherished
objects, nor plotting against men who are in possession of that on
which you set your heart. The man who lacks any of these things must,
of necessity, be distracted, and be for ever complaining against the
Gods. But reverence and respect for your own intelligence will bring
you to agreement with yourself, into concord with mankind, and into
harmony with the Gods, whom you will praise for all their good gifts
and guidance.

17. Upward, downward, round and round run the courses of the
elements. But the course of virtue is like none of these; it follows a
diviner path, well-directed in a way that is hard for us to
understand.

18. Strange are the ways of men! They can speak no good word of the
contemporaries with whom they live; yet they count it a great thing to
gain the praises of a posterity whom they never saw nor shall see. As
well might we grieve because we cannot hear the praises of our
ancestors.

19. If a thing seems to you very difficult to accomplish, conclude not
that it is beyond human power. But, if you see that anything is within
man's power, and part of his proper work, conclude that you also may
attain to it.

20. In the gymnasium, if some one scratches us with his nails, or in a
sudden onset bruises our head, we express no resentment; we are not
offended; nor do we suspect him for the future as one who is plotting
against us. We are on our guard against him, it is true, but not as
against an enemy or a suspected person. In all good humour we simply
keep out of his way. Let us thus behave in other affairs of life, and
overlook the many injuries which are done to us, as it were, by our
antagonists in the gymnasium of the world. As I said, we may keep out
of their way, but without suspicion or hatred.

21. If any one can convince or shew me that I am wrong in thought or
deed, I will gladly change. It is truth that I seek; and truth never
yet hurt any man. What does hurt is persistence in error or in
ignorance.

22. I do my duty, and for the rest am not distracted by anything which
is inanimate or irrational, or which has lost or ignores the proper
way.

23. Use the brute creation, and also all material things, in the
spirit of magnanimity and freedom which becomes him who has reason in
using that which has it not. Towards men, who have reason, act in a
social spirit. In every business call the Gods to aid thee, nor
trouble how long this business shall endure; three hours spent therein
may suffice you.

24. Alexander of Macedon and his muleteer, when they died, were in a
like condition. They were either alike resumed into the seminal source
of all things, or alike dispersed among the atoms.

25. Consider all the many things, both physical and spiritual, that
are adoing within each of us at the very same instant of time; and you
will wonder the less at the far greater multitudes of things, even all
that is, which exist together in the one-and-all which we call the
Universe.

26. Should some one ask you how the name Antoninus is written, would
you not carefully pronounce to him each one of the letters? Should he
then begin an angry dispute about it, would you also grow angry, and
not rather mildly count over the several letters to him? Thus in life
remember that each duty is made up of a number of elements. We should
observe all these calmly; and, without anger at those who are angry
with us, we should set about accomplishing the task which lies before
us.

27. Is it not cruel to restrain men from pursuing what appears to be
their own advantage? And yet, in a manner, you deny them this liberty
when you shew anger at their errors. Men are assuredly attracted to
what seems to be their own advantage. ``Yes,'' you say, ``but it is not
their advantage.'' Instruct them, then, and make this evident to them,
but without anger.

28. Death is the cessation of the sensual impressions, of the impulses
of the passions, of the questionings of reason, and of the servitude
to the flesh.

29. It is shame and dishonour that, in any man's life, the soul should
faint from its duty while the body still holds out.

30. See to it that you fall not into Caesarism: avoid that stain, for
it may come to you. Guard your simplicity, your goodness, your
sincerity, your dignity, your reticence, your love of justice, your
piety, your kindliness, your affection for your kin, and your
constancy to your duty. Endeavour earnestly to continue such as
philosophy would make you. Reverence the Gods, and help mankind. Life
is short, and the one fruit of it in this world is a pure mind and
unselfish conduct. Be in all things the disciple of Antonine. Imitate
his resolute constancy to rational action, his level equability, his
godliness, his serenity of countenance, his sweetness of temper, his
contempt of vainglory, his keen attempts to comprehend things. Remember
how he never quitted any subject till he had thoroughly examined it
and understood it, and how he bore with those who blamed him unjustly,
without making any angry retort: how he was never in a hurry; how he
discouraged calumny; how closely he scanned the manners and actions of
men; how cautious he was in reproaching any man; how free from fear,
suspicion, or sophistry; how little contented him in the matter of
house, furniture, dress, food, servants; how patient he was of labour,
and how slow to anger. So abstemious was his life that he could hold
out until evening without relieving himself, except at the usual
hour. What a firm and loyal friend he was; how patient of frank
opposition to his opinions; how glad if any one could set him right!
How religious he was, and yet how free from superstition! Follow in
his steps that your last hour may find you with a conscience as easy
as his.

31. Sober yourself, recall your senses. Shake sleep from you, and know
that it was a dream that troubled you; and, now that you are broad
awake again, regard the waking world as you did the dream.

32. I am made up of a frail body and a soul. To the body all things
are indifferent, because it cannot distinguish them; and to the mind
all things are indifferent also which arise not from its own
activities. All these are indeed in its own power, but it is concerned
with only such of them as are present. Its past and future activities
are indifferent to it now.

33. No toil for hand or foot is against Nature, so long as it is
proper for hand or foot to do. No more, then, is toil contrary to the
nature of man, as man, so long as he is doing work appointed for man
to do; and if it be not contrary to his nature it cannot be evil for
him.

34. How many are the pleasures that have been enjoyed by robbers,
rakes, parricides, and tyrants!

35. Do you not see how common artificers, though they may humour the
public to a certain extent, cling to the rules of their art, and
cannot endure to depart from them? Is it not grievous, then, that the
architect and the physician should shew greater respect for the rules
of their several professions, than man shews for his own reason, which
he possesses in common with the Gods?

36. Asia and Europe are mere corners of the Universe: the whole sea is
but a drop, Athos a clod. All the present is but an instant in
eternity. All things are small, changeable, and fleeting. Everything
proceeds from the universal intelligence, either directly or as a
consequence. Thus, the jaws of lions, poisons, all evil things such as
thorns or mire, are the consequences of the grand and the
beautiful. Do not, then, imagine that they are foreign to that which
you revere, but consider well the source of all things.

37. He who has seen the present has seen all that either has been from
all eternity, or will be to all eternity, for all things are alike in
kind and form.

38. Consider frequently the connexion of all things in the Universe,
and their relation to each other. All things are in a manner
intermingled with one another, and are, therefore, mutually
friendly. For one thing comes in due order after another, by virtue of
local movements, and of the harmony and unity of the whole.

\newpage

39. Adapt yourself to the things which your destiny has given you:
love those with whom it is your lot to live, and love them with
sincere affection.

40. A tool, an instrument, a utensil, is in good case when it is fit
for its proper work: yet its maker remains not by it. But within the
organisms of Nature there remains and resides the power which made
them. You ought, therefore, to reverence this power the more,
believing that if you act in deference to its will, all will happen to
you in reason; for so in reason the Universe ranges all.

41. Whenever we imagine that anything which lies not in our power is
good or evil for us, if the evil befall us or if we miss the good, we
inevitably blame the Gods, and hate the men who are, or whom we
suspect to be, the cause of our disaster or our loss. Our solicitude
about such things leads to much injustice; but if we judge only the
things that are in our power to be good or evil, there is no reason
left for accusing the Gods or for hating men.

42. We are all co-operating in one great work, some with knowledge and
understanding, others ignorantly and without design. It is in this
sense, I think, that Heraclitus says that men are working even while
they sleep, working together in all that is being done in the
Universe. Each works in a different way; and even those contribute
abundantly who murmur and try to oppose and to frustrate the course of
nature. The world has need even of such as these. It remains then for
you to make sure which is the class in which you rank yourself. The
presiding mind will assuredly use you to good purpose one way or
other; and will enlist you among its labourers and fellow-workers. But
see to it that the part that falls to you lie not in the vulgar comic
passage of the play, of which Chrysippus has spoken.

43. Does the sun pretend to perform the work of the rain, or
Aesculapius that of Ceres? What of the several stars? Are they not
different, yet all jointly working for the same end?

44. If the Gods took counsel about me and what should befall me,
doubtless then-counsel was good. It is difficult to imagine Gods
wanting in forethought, and what could move them to do me wilful harm?
What advantage would thence accrue, either to themselves or to the
Universe which is their special care? If they have not taken counsel
about me in particular, they certainly have done so about the common
interest of the Universe, and I therefore should accept cheerfully and
contentedly the fate which is the outcome of their ordinance. If,
indeed, they take no counsel about anything (which it were impious to
believe), then let us quit our sacrifices, our prayers, and our oaths,
and all acts of devotion which we now perform as if they lived and
moved amongst us. But, granting that the Gods take no thought for my
affairs, I may still deliberate about myself. It is my business to
consider my own interest. Now, each man's interest is that which
agrees with the structure of his nature, and my nature is rational and
social. As Antoninus, my city and my country is Rome; as a human being
it is the world. That alone, then, which profits these two cities can
profit me.

45. All that happens to the individual is of profit to the whole. This
would suffice. But if you consider closely you will see that it is
also a general truth that all that happens to one man is of profit to
the rest of mankind. ``Profit'' here should be taken in a somewhat
general sense, as referring to things indifferent.

46. In the amphitheatre and other such resorts the same or similar
spectacles, continually presented, cloy at last. It is even so in all
our experience of life. All things, first and last, are alike, and
like derived. When shall the end be?

47. Think continually of all the men that are dead and gone, men of
every sort and condition, of all manner of pursuits, and of every
nation. Return back to Philistion, Phoebus, and Origanion. Pass down
to other generations of the dead. We must all change our habitation
and go to that place whither so many great orators, so many venerable
philosophers, Heraclitus, Pythagoras, Socrates, and so many heroes
have gone before, and so many generals and princes have followed. Add
to these Eudoxus, Hipparchus, Archimedes, and other keen, great,
laborious, cunning and arrogant spirits; yea, such as have wittily
derided this fading mortal life which is but for a day, as did
Menippus and his brethren. Consider that all these are long since in
their graves. And wherein here is the harm for them; or even for men
whose names are not remembered? The one precious thing in life is to
spend it in a steady course of truth and justice, with kindliness even
for the false and the unjust.

48. When you would cheer your heart, consider the several excellencies
of those that live around you. Consider the activity of one, the
modesty of another, the generosity of a third, and the other virtues
of the rest. Nothing rejoices the heart so much as instances, the more
the better, of goodness manifested in the characters of those around
us. Let us, therefore, have such instances ever present for
reflection.

49. Are you grieved that you weigh only these few pounds, and not
three hundred? If not, is there greater reason to sorrow if you live
only so many years and no longer? You are satisfied with your allotted
quantity of matter; content yourself then likewise with the span of
time appointed you.

50. Try to persuade men to agree with you; but whether they agree or
not, pursue the course you have marked out when the principles of
justice point that way. Should one oppose you by force, act with
resignation, and shew not that you are hurt, use the obstruction for
the exercise of some other virtue, and remember that your purpose
involved the reservation that you were not to aim at impossibilities.
What, after all, was your aim? To make some good effort such as
this. Well, then, you have succeeded, even though your first purpose
be not accomplished.

51. The vain-glorious man places his happiness in the action of
others. The sensualist finds it in his own sensations. The wise man
realizes it in his own work.

52. You have it in your power to form no opinion about this or that,
and so to have peace of mind. Things material have no power to form
our opinions for us.

53. Accustom yourself to attend closely to what is said by others, and
as far as possible to penetrate into the mind of the speaker.

54. What profits not the swarm profits not the bee.

\newpage

55. If the sailors revile their pilot, or the sick their physician,
whom will they follow or obey? And how will the one secure safety to
the crew, or the other health to the patients?

56. How many who entered the world with me are already departed!

57. To the jaundiced, honey seems bitter; and water is a thing of
dread to those bitten by mad dogs. To boys a ball is a glorious
thing. Why, then, am I angry? Has error in the mind less power than a
little bile in the jaundiced, or a little poison in him who is bitten?

58. No man can prevent you from living according to the plan of your
nature; and nothing can befall you which is contrary to the plan of
the nature of the Universe.

59. Consider what men are; whom they seek to please; what they expect
to gain, and how they go about to compass their ends. Think how soon
eternity will shroud all things, and how much is already shrouded.

\terminus{}

\chapter[What is vice?]{}

1. \textsc{What is vice?} It is what you have often seen. In every
instance of it keep in mind that you have often seen the like before.
Search up and down; you will find sameness everywhere. Among the events
which fill the history of ancient, middle, and present ages; among the
things of which our cities and our households are full to-day, nothing
is new, all is familiar and fleeting.

2. How can the great principles of life become dead if the impressions
which correspond to them be not extinguished? These impressions you
may still rekindle. I can always form the proper opinion of this or
that; and, if so, why am I disturbed? What is external to my mind is
of no consequence to it. Learn this, and you stand upright; you can
always renew your life. See things again as once you saw them, and
your life is made new again.

3. Your vain concern for shows, for stage plays, for flocks and herds,
your little combats, are as bones cast for the contention of puppies,
as baits dropped into a fishpond, as the toil of ants and the burdens
that they bear, as the scampering of frightened mice, or the antics of
puppets jerked by wires. It is then your duty amid all this to stand
firm, kindly and not proud, yet to understand that a man's worth is
just the worth of that which he pursues.

4. In conversation we should give good heed to what is said, and in
every enterprise we should attend to what is done. In the latter case,
at once look to the end in view, and, in the former, note the meaning
intended.

5. Is my understanding sufficient for this business or not? If it be
sufficient, I use it for the work in hand as an instrument given to me
by nature. If it be not sufficient, I either give place to one better
fitted for the achievement, or, if for some reason this be not a
proper course, I do it as best I can, taking the aid of those who, by
directing my mind, can accomplish something fit and serviceable for
the common good. For all that I do, whether by myself or with the help
of others, should be directed solely towards what is fit and useful
for the public service.

6. How many of those who were once so mightily acclaimed are delivered
up to oblivion! And how many of those who acclaimed them are dead and
gone this many a day!

7. Be not ashamed of taking assistance. It is laid upon you to do your
part, as on a soldier when the wall is stormed. What, then, if you are
lame, and cannot scale the battlements alone, but can with another's
help?

8. Be not troubled about the future. You will come to it, if need be,
with the same power to reason, as you use upon your present business.

9. All things are twined together, in one sacred bond. Scarce is there
one thing quite foreign to another. They are all ranged together, and
leagued to form the same ordered whole. The Universe, compact of all
things, is one; through all things runs one divinity; being is one;
and law, which is the reason common to all intelligent creatures; and
truth is one as well, that is if there be but one sort of perfection
possible to all beings which are of the same nature and partake of the
same rational power.

10. Everything material is soon engulfed in the matter of the whole,
and every active cause is swiftly resumed into the Universal
reason. The memory of all things is quickly buried in eternity.

11. In the reasoning being to act according to nature is to act
according to reason.

12. Be upright either by nature or by correction.

13. In an organic unity bodily members play the same part as reasoning
beings among separate existences, since both are fitted for one joint
operation. This thought will come home to you the more vividly if you
say often to yourself: ``I am a member of the mighty organism which is
made up of reasoning beings.'' If, instead of a member, you say that
you are merely a part, you have not as yet attained to a heartfelt
love of mankind. As yet you love not well-doing for its own sake
alone, and you still perform your bare duty, with no thought that you
are your own benefactor by the deed.

14. From the world without let what will affect whatever parts are
subject to such affection. Let the part which suffers complain, if it
will, of the suffering. But I, if I admit not that the hap is evil,
remain uninjured. Not to admit it is surely in my power.

\newpage

15. Let any one say or do what he pleases, I must be a good man. It is
just as gold, or emeralds, or purple might say continually: ``Let men
do or say what they please, I must be an emerald, and retain my
lustre''.

16. The soul which rules you vexes not itself. It does not, for
example, awake its own fears or arouse its own desires. If another can
raise grief or terror in it, let him do so. By its own impressions it
will not be led into such emotions.

Let the body take thought, if it can, for itself, lest it suffer
anything, and complain when it suffers. The soul, by means of which we
experience fear and sorrow, and by means of which, indeed, we receive
any impression of these, will admit no suffering. You cannot force it
to any such opinion.

The ruling part is, in itself, free from all dependence, unless it
makes itself dependent. Similarly, it may be free from all disturbance
and obstruction, if it does not disturb and obstruct itself.

17. To have good fortune is to have a good spirit, or a good
mind. What do you here, Imagination? Be gone, I say, even as you
came. I have no need for you. You came, you say, after your ancient
fashion: I am not angry with you, only, be gone!

18. Do you dread change? What can come without it? What can be pleasanter
or more proper to universal nature? Can you heat your bath unless wood
undergoes a change? Can you be fed unless a change is wrought upon
your food? Can any useful thing be done without changes? Do you not
see, then, that this change also which is working in you is even such
as these, and alike necessary to the nature of the Universe?

\newpage

19. Through the substance of the Universe, as through a torrent, all
bodies are borne. They are all of the same nature, and fellow-workers
with the whole, even as our several members are fellow-workers with
one another. How many a Chrysippus, how many a Socrates, how many an
Epictetus hath the course of ages swallowed up! Let this thought be
with you about every man, and upon all occasions.

20. For this alone I am concerned; that I do nothing that suits not
the nature of man, nothing as man's nature would not have it, nothing
that it wishes not yet.

21. The time is at hand when you shall forget all things, and when all
shall forget you.

22. It is man's special business to love even those who err; and to
this love you attain, if it is borne in upon you that even these
sinners are your kin, and that they offend through ignorance and
against their will. Remember also that in a little while both you and
they must die: remember before all things that they have not harmed
you, for they have not made your soul worse than it was before.

23. Presiding nature from the universal substance, as from wax, now
forms a horse, now breaks it up again, making of its matter a tree,
afterwards a man, and again something different. Each of these shapes
subsists but for a little. Yet there is nothing dreadful for the chest
in being taken to pieces, any more than there formerly was in being
put together.

24. A wrathful look is completely against nature. When the countenance
is often thus deformed, its beauty dies, in the end is quenched for
ever, and cannot be revived again. Seek to comprehend from this very
fact that it is against reason. And if the sense of moral evil be gone
as well, why should a man wish to remain alive?

25. In a little space Nature, the supreme and universal ruler, will
change all things that you behold; out of their substance she will
make other things, and others again out of the substance of these, so
that the Universe may be ever new.

26. Whenever someone offends you, consider straightway how he has
erred in his conceptions of good or evil. When you see where his error
lies you will pity him, and be neither surprised nor angry. Indeed you
yourself perhaps still wrongly count good the same things as he does,
or things just like them. Your duty then is to forgive. And, if you
cease from these false ideas of good and bad, you will find it the
easier to grant indulgence to him who is still mistaken.

27. Dwell not on what you lack so much as on what you have
already. Select the best of what you have, and consider how
passionately you would have longed for it had it not been yours. Yet
be watchful, lest by this joy in what you have you accustom yourself
to value it too highly; so that, if it should fail, you would be
distressed.

28. Retire within yourself. The reasoning power that rules you
naturally finds contentment with itself in just dealing, and in the
calm which such dealing brings.

29. Blot out imagination. Check the brutal impulses of the
passions. Confine your energies to the present time. Observe clearly
all that happens either to yourself or to another. Divide and analyse
all objects into cause and matter. Take thought for your last
hour. Let another's sin remain where the guilt lies.

\newpage

30. Apply your mind to what is said. Penetrate all happenings and the
causes thereof.

31. Rejoice yourself with simplicity, modesty, and indifference to all
things that lie between good and bad. Love mankind, and obey God. ``All
things,'' says someone, ``go by law and order.'' But what if there be
naught beyond the atoms? Even if that be so, suffice it to remember
that all things, save very few, are swayed by law.

32. Concerning death: If the Universe be a concourse of atoms, death
is a scattering of these; if it be an ordered unity, death is an
extinction or a translation to another state.

33. Concerning pain: Pain which cannot be borne brings us
deliverance. Pain that lasts musts needs be bearable. The mind can
abstract itself from the body, and the soul takes no hurt. As to the
parts which suffer by pain, let them, if they can, make their own
protest.

34. Concerning glory: Consider the understanding of men, what they
shun, and what they pursue. And reflect that, as heaps of sand are
driven one upon another, and the later drifts bury and hide those that
went before, so, too, in life the former ages are soon buried by the
next.

35. This from Plato: ```To the man who has true grandeur of mind, and
who contemplates all time and all being, can human life appear a great
matter? `Impossible,' says the other. `Can then such a one count death
a thing of dread?' `No, indeed.'''

36. It is a saying of Antisthenes, that it is the part of a king to do
good and reap reproach.

37. It is a shameful thing that the countenance should obey the mind,
should compose and order itself as the mind bids it, while the mind
cannot compose and order itself as it wills.

38. Vain is all anger at external things \\
    For they regard it nothing.--

39. Give joy to us and to the immortal Gods.

40. For life is, like the laden ear, cut down; \\
    And some must fall and some unreaped remain.

41. Me and my children, if the Gods neglect, \\
    It is for some good reason.

42. For I keep right and justice on my side.

43. Weep not with them, and still these throbs of woe.

44. From Plato:--``I would make him this just answer, `You are
mistaken, my friend, to think that a man of any worth should count the
chances of living and dying. Should he not rather, in all he does,
consider simply whether he is acting justly or unjustly, whether he is
playing the part of a good man or a bad?'''

45. He says again:--``In truth, Athenians, the matter stands thus:
Wheresoever a man has chosen his stand, judging it the fittest for
him, or wheresoever he is stationed by his commander, there, I think,
he should stay at all hazards, making no account of death, or any
other evil but dishonour''.

46. Again:--``Consider, my friend, whether the truly noble and the
truly good be not something quite apart from saving and being
saved. The man who is a man indeed should not set his heart on living
through a few more years of life, nor should he make that the end of
his desire. Rather he should commit the matter to the will of God;
assenting to the maxim which even women use, that `no man can elude
his destiny,' and studying in addition how he may spend the life that
remains to him for the best''.

47. Contemplate the courses of the stars, as one should do that
revolves along with them. Consider also without ceasing the changes of
elements, one into another. Speculations upon such things cleanse away
the filth of this earthly life.

48. It is a good thought of Plato's, that when we discourse of men we
should ``look down, as from a high place,'' on all things earthly; on
herds and armies; on husbandry and marriage; on partings, births, and
deaths; on the tumults of the courts of justice; on the desert places
of the earth; on the varied spectacle of savage nations; on feasting
and lamentation; on traffic; on the medley of all things, and the
order which emerges from their contrariety.

49. Consider the past, and the revolutions of so many Empires; and
thence you may foresee what shall happen hereafter. It will be ever
the same in all things; nor can events leave the rhythm in which they
are now moving. Wherefore it is much the same to view human life for
forty, as for a myriad of years. What more is there to see?

50. To earth returns whatever sprang from earth, \\
    But what's of heavenly seed remounts to heaven.

This imports either the loosing of a knot of atoms, or a similar
dispersion of immutable elements.

51. By meats and drinks, and charms and magic arts \\
    Death's course they would divert, and thus escape. \\
    .  .  .  .  .  .  . \\
    The gale that blows from God we must endure, \\
    Toiling, but not repining.....

52. He is a better wrestler than you, but not more public-spirited,
more modest, or better prepared for the accidents of fate; not more
gentle toward the short-comings of his neighbours.

53. Wherever we can act conformably to the reason which is common to
Gods and men, there we have nothing to dread. Where we can profit by
prosperous activity which proceeds in agreement with the constitution
of our nature, we need suspect no harm.

54. In all places, and at all times you may devoutly accept your
present fortune, and deal in justice with your present company. You
may take pains to understand all arising imaginations, that none may
steal upon you before you comprehend them.

55. Pry not into the souls of others; but rather look straight to the
goal whither nature is leading you; whither the nature of the Universe
by external events, and whither your own nature by the tendency of
your own action. Each being must perform the part for which it was
created. Now all other beings are created for the sake of those among
them which have reason; as all lower things exist for the sake of
things superior to them; and reasoning beings were created for one
another. The leading principle in man's nature, then, is the social
spirit; and the second is victory over the solicitations of the
body. For it is proper to the workings of reason to set bounds to
themselves, and never to be overpowered by the calls of sense or by
the stirrings of passion, both of which are animal in their
nature. The intellect claims to reign over these, and never to be
subjected to them; and rightly, because it is equipped to command and
use all the lower powers. The third element in the constitution of a
reasoning being is caution against rashness and error. Let the soul go
forth straight upon her way in the possession of these principles, and
she stands seized of her full estate.

56. Consider yourself as dead, your life as finished and past. Live
what yet remains according to Nature's laws, as an overplus granted to
you beyond your hope.

57. Love that only which is your hap, which comes upon you as your
part in Fate's great spinning. What, indeed, can fit you better?

58. Upon every accident keep in view those to whom the like has
happened. They stormed at the event, wondered and complained. But now
where are they? They are gone for ever. Why should you act the like
part? Leave these unnatural commotions to fickle men who change and
are changed. Yourself take thought how you may make good use of such
events. Good use for them there is; they will make matter for good
actions. Let it be your sole effort and desire to gain your own
approval in every action; and remember that the material objects of
both that effort and of that desire are things indifferent.

59. Look inward. Within is the fountain of Good. Dig constantly and it
will ever well forth.

60. Keep the body steady, without irregularity, whether in its motions
or in its postures. For, as the soul shews itself in the countenance
by a wise and graceful air, it should require the same expressive
power of the whole body. But all this must be practised without
affectation.

61. The art of Life is more like that of the wrestler than of the
dancer; for the wrestler must always be ready on his guard, and stand
firm against the sudden, unforeseen efforts of his adversary.

62. Consider constantly what manner of men they are whose approbation
you desire, and what may be the character of their souls. Then you
will neither accuse such as err unwillingly, nor need their
commendation when you look into the springs of their opinions and
their desires.

63. ``Every soul,'' says Plato, ``parts unwillingly with truth.'' You may
say the same of justice, temperance, good-nature, and every virtue. It
is most necessary to keep this ever in mind; for, if you do, you will
be more kindly towards all men.

64. In all pain keep in mind that there is no baseness in it, that it
cannot harm the soul which guides you, nor destroy that soul as a
reasoning or as a social force. In most pain you may find help in the
saying of Epicurus, that ``pain is neither unbearable nor everlasting,
if you bear in mind its narrow limits, and allow no additions from
your imagination.'' Remember also that we are fretted, though we see it
not, by many things which are of the same nature as pain, things such
as drowsiness, excessive heat, want of appetite. When any of these
things annoy you, say to yourself that you are giving in to pain.

65. Look to it that you feel not towards the most inhuman of mankind,
as they feel towards their fellows.

66. Whence do we conclude that Telauges had not a brighter genius than
Socrates? 'Tis not enough that Socrates died more gloriously or argued
more acutely with the sophists; or that he kept watch more patiently
through a frosty night; or because, when ordered to arrest the
innocent Salaminian, he judged it more noble to disobey; or because of
any stately airs and graces he assumed in public, in which we may very
justly refuse to believe. But, assuming all this true, when we
consider Socrates, we must ask what manner of soul he had. Could he
find contentment in acting with justice towards men, and with piety
towards the Gods, neither vainly provoked by the vices of others, nor
servilely flattering them in their ignorance; counting nothing strange
that the Ruler of the Universe appointed, not sinking under anything
as intolerable, and never yielding up his soul in surrender to the
passions of the flesh.

67. Nature has not so blended the soul with the body that it cannot
fix its own bounds, and execute its own office by itself. It is very
possible to be a God among men, and yet be recognised by
none. Remember that always, and this as well, that the happiness of
life lies in very few things. And though you despair of becoming great
in Logic or in Science, you need not despair of becoming a free man,
full of modesty and unselfishness, and of obedience unto God.

68. It is in your power to live superior to all violence, and in the
greatest calm of mind, were all men to rail against you as they
pleased; and though wild beasts were to tear asunder the wretched
members of this fleshly mass which has grown with your growth. What is
to hinder the soul amid all this from preserving itself in all
tranquillity, in just judgments about surrounding things, and in ready
use of whatever is cast in its way? Judgment may say to accident:--
``Your real nature is this or that, though you appear otherwise in the
eyes of men.'' Use may say to circumstance:--``I was looking for
you. To me all that is present is ever matter for rational and social
virtue, in sum, for that art which is proper both to man and God. All
that befalls is fit and familiar for the purposes of God or
man. Nothing is either new or intractable, but everything is well
known and fit to work upon''.

69. It is the perfection of morals to spend each day as if it were the
last of life, without excitement, without sloth, and without
hypocrisy.

70. The Gods, who are immortal, are not vexed that in a long eternity
they must ever bear with the wickedness and the multitude of
sinners. Nay, they even lavish on them all manner of loving care. But
you, who are presently to cease from being, can, forsooth, endure no
more, though you are one of the sinners yourself!

71. It is ridiculous that you flee not from the vice that is in
yourself, as you have it in your power to do; but are still striving
to flee from the vice in others, which you can never do.

72. Whatever the rational and social faculty finds fit neither for
rational nor for social ends, it justly ranks as inferior to itself.

73. When you have done a kind action, another has benefited. Why do
you, like the fools, require some third thing in addition--a
reputation for benevolence or a return for it.

74. No man wearies of what brings him gain, and your gain lies in
acting according to nature. Be not weary, therefore, of gaining by the
act which gives others gain.

\newpage

75. Nature set about making an ordered universe; and now, either all
that is follows a law of necessary consequence and connexion, or we
must admit that there is least rationality in the things which are
most excellent, and which appear to be most special objects for the
impulses of the universal mind. Remembrance of this will give you
calmness on many an occasion.

\terminus{}
\chapter[For repressing vain glory,...]{}

1. \textsc{For repressing vain glory,} it serves to remember that it
is no longer in your power to make your whole life, even from your
youth onwards, a life worthy of a philosopher. It is known to many,
and you yourself know also, how far you are from wisdom. Confusion
is upon you, and it now can be no easy matter for you to gain the
reputation of a philosopher. The conditions of your life are against
it. Now therefore, as you see how the matter truly lies, put from you
all thoughts of reputation among men; and let it suffice you to live
so long as your nature wills, though that be but the scanty remnant of
a life. Study, therefore, the will of your nature, and be solicitous
about nothing else. You have made many efforts and wandered much, but
you have nowhere found happiness; not in syllogisms, not in riches, not
in fame or pleasure, not in anything. Where, then, is it? In acting
that part which human nature requires. How can you act that part? By
holding principles as the source of your desires and actions. What
principles? The principles of good and evil: That nothing is good for a
man which does not make him just, temperate, courageous, and free; and
that nothing can be evil which tends not to make him the contrary of all
these.

2. Upon every action ask yourself, what is the effect of this for me?
Shall I never repent of it? I shall presently be dead, and all these
things gone. What more should I desire if my present action is
becoming to an intelligent and a social being, subject to the same law
with Gods?

3. Alexander, Caesar, Pompey, what were they compared with Diogenes,
Heraclitus, Socrates? These knew the nature of things, their causes
and their matter, and the minds within them were at one. As to the
former, how many things they schemed for, and to how many were they
enslaved!

4. Men will go their ways none the less, though you burst in protest.

5. Before all things, be not perturbed. Everything comes to pass as
directed by universal Nature, and in a little time you will be
departed and gone, like Hadrianus and Augustus. Then, scan closely the
nature of what has befallen, remembering that it is your duty to be a
good man. Do unflinchingly whatever man's nature requires, and speak
as seems most just, yet in kindliness, modesty, and sincerity.

6. It is Nature's work to transfer what is now here into another
place, to change things, to carry them hence, and set them
elsewhere. All is change, yet is there no need to fear innovation, for
all obey the laws of custom, and in equal measure all things are
apportioned.

7. For every nature it is sufficient that it goes on its way, and
prospers. The rational nature prospers while it assents to no false or
uncertain opinion, while it directs its impulses to unselfish ends
alone, while it aims its desires and aversions only at the things
within its power, and while it welcomes with contentment all that
universal Nature ordains. The nature of each of us is part of
universal Nature, as the leaf is part of the tree; the leaf, indeed,
is part of an insensible and unreasoning system which can be
obstructed in its workings; but human nature is part of that universal
system which cannot be impeded, and which is intelligent and
just. Hence is meted out, suitably to all, our proper portions of
time, of matter, of active principle, of powers, and of events. Yet
look not to find that each several thing corresponds exactly with any
other. Consider rather the whole nature and circumstances of the one,
and compare them with the whole of the other.

8. You lack leisure for reading; but leisure to repress all insolence
you do not lack. You have leisure to keep yourself superior to
pleasure and pain and vain glory, to restrain all anger against the
ungrateful, nay, even to lavish loving care upon them.

9. Let no man any more hear you railing on the life of the court; nay,
revile it not to your own hearing.

10. Repentance is a self-reproving, because we have neglected
something useful. Whatever is good must be useful in some sort, and
worthy of the care of a good and honourable man. Now, such a man could
never repent of neglecting some opportunity of pleasure. Pleasure,
then, is neither useful nor good.

11. Of each thing ask: What is this in itself and by its constitution?
What is its substance or matter? What is its cause? What is its
business in the Universe? How long shall it endure?

\newpage

12. When you are reluctant to be roused from sleep, remember that it
accords with your constitution and with human nature to perform social
actions. Sleep is common to us with the brutes. Now, whatever accords
with the nature of each species must be most proper, most fitting, and
most delightful to it.

13. Constantly, and, if possible, on every occasion, apply to your
imaginations the methods of Physics, Ethics, and Dialectic.

14. Whomsoever you meet, say straightway to yourself:--What are this
man's principles of good and evil? For if he holds this or that
doctrine concerning pleasure and pain, and the causes thereof,
concerning glory and infamy, death and life, it will seem to me
neither strange nor wondrous that this or that should be his
conduct. I shall bear in mind that he has no choice but to act so.

15. Remember that, as 'tis folly to be surprised that a fig-tree bears
figs, so is it equal folly to be surprised that the Universe produces
those things of which it was ever fruitful. It is folly in a physician
to be surprised that a man has fallen into a fever; or in a pilot that
the wind has turned against him.

16. Remember that to change your course, and to follow any man who can
set you right is no compromise of your freedom. The act is your own,
performed on your own impulse and judgment, and according to your own
understanding.

17. If the doing of this be in your own power, why do it thus? If it
be in another's, whom do you accuse? The atoms or the Gods? To accuse
either is a piece of madness. Therefore accuse no one. Set right, if
you can, the cause of error; if you cannot, correct the result at
least. If even that be impossible, what purpose can your accusations
serve?  Nothing should be done without a purpose.

18. That which dies falls not out of the Universe. If then it stays
here, here too it suffers a change, and is resolved into those
elements of which the world, and you too, consist. These also are
changed, and murmur not.

19. The horse, the vine--all things are formed for some purpose. Where
is the wonder? Even the sun saith, ``I was formed for a certain work'';
and similarly the other Gods. For what end are you formed? For
pleasure?  Look if your soul can endure this thought.

20. Nature has an aim in all things, in the end and surcease of them
no less than in their beginning and continuance. It is even as a man
casting a ball. Where, then, is the good for the ball in its rising;
where the harm in dropping; where even is the harm when it has fallen
down? Where is the bubble's good while it holds together, where is the
evil when it is broken? So it is with the lamp which now burns and
anon goes out.

21. Turn out the inner side of this body, and view it as it is. What
shall it become when it grows old, or sickly, or decayed? The praiser
and the praised, the rememberer and the remembered are of short
continuance, and that in a mere corner of this narrow region, where,
narrow though it be, men cannot live in concord, no, not even with
themselves. And yet the whole world is but a point.

22. Attend well to what is before you, whether it be a principle, an
act, or a word. This your suffering is well merited, for you would
rather become good to-morrow than be good to-day.

23. Am I doing aught? Let me do it in a spirit of service to
mankind. Does aught befall me? I accept it and refer it to the Gods,
the universal source from which come all things in the chain of
consequence.

24. The accompaniments of bathing: oil, sweat, filth, foul water--how
nauseous are they all! Even so is every part of life, and everything
that meets us.

25. Lucilla buried Verus, and soon followed him to the grave. Secunda
saw the death of Maximus, and soon herself died. Epitynchanus buried
Diotimus, and then Epitynchanus was buried. Antoninus mourned
Faustina, and thereafter Antoninus was mourned. Celer buried Hadrian,
and then Celer was buried. All go the same way. The cunning men who
foretold the fates of others, or who swelled with pride--where are
they now? Where are these keen wits, Charax, and Demetrius the
Platonist, and Eudaemon, and their like? All were for a day, and are
long dead and gone; some scarce remembered even for a little after
death; some turned to fables; some faded even from the memory of
tales. Wherefore remember this: either the poor mixture which is you,
must be dispersed, or the faint breath of life must be quenched, or
removed and brought into another place.

26. The joy of man is to do his proper business. And his proper
business is to be kindly to his fellows, to rise above the stirrings
of sense, to be critical of every plausible imagination, and to
contemplate universal Nature and all her consequences.

27. We have all of us three relations: the first to the manifold
occasions of our state; the second to the supreme divine cause from
which proceed all things unto all men; the third to those with whom we
live.

28. Pain is either an evil to the body; and then let the body so
declare it; or an evil to the soul. But the soul can maintain her own
serenity and calm; and refuse to conceive pain as an evil. All
judgment, intention, desire and aversion are within the soul, to which
no evil can ascend.

29. Blot out false imaginations, and say often to yourself:--It is
now in my power to preserve my soul free from all wickedness, all
lust, all confusion or disturbance. And then, as I truly discern the
nature of things, I can use them all in due proportion. Be ever
mindful of this power which Nature has given you.

30. Speak, whether in the Senate or elsewhere, with dignity rather
than elegance; and let your words ever be sound and virtuous.

31. The court of Augustus, his wife, his daughter, his descendants and
his ancestors; his sister, and Agrippa; his kinsmen, familiars and
friends; Areius and Maecenas; his physicians and his flamens--death
has them all. Think next of the death of a whole house, such as
Pompey's, and of what we meet sometimes inscribed on tombs: He was the
last of his race. Last of all, consider the solicitude of the
ancestors of such men to leave a succession of their own
posterity. Yet, at the end, one must come the last, and with him dies
all that house.

32. Order your life in its single acts, so that if each, as far as may
be, attains its end, it will suffice. In this no one can hinder
you. But, you say, may not something external withstand me?--Nothing
can keep you from justice, temperance, and wisdom.--Yet, perhaps
some other activity of mine may be obstructed.--True, but by
yielding to this impediment, and by turning with calmness to that
which is in your power, you may happen on another course of action
equally suited to the ordered life of which we are speaking.

33. Receive the gifts of fortune without pride; and part with them
without reluctance.

34. You have seen a hand, a foot, or a head, cut off from the rest of
the body, and lying dead at a distance from it. Even such as these
does he make himself, so far as he can, who repines at what befalls,
who severs himself from his fellow-men, or who does any selfish
deed. Are you cast forth from the natural unity? Nature made you to be
a part of the whole, but you have cut yourself off from it. Yet here
there is the glorious provision that you may re-unite yourself if you
will. In no other case has God granted the privilege of re-union to a
separated or severed part. Yet behold the goodness and bounty with
which God hath honoured mankind. He first puts it in their power not
to be severed from this unity; and then, even when they are thus
severed, he suffers them to return once more, to take their places as
parts of the whole, and to grow one with it again.

35. Universal Nature, as she has imparted to each rational being
almost all its faculties and powers, has given to us this one in
particular among them. As Nature converts to her use, ranges in
destined order, and makes part of herself all that withstands or
opposes her; so each rational being can make every impediment in his
way a proper matter for himself to act upon, and can use it for his
guiding purpose, whatever it may be.

36. Do not confound yourself by considering the whole of life, and by
dwelling upon the multitude and greatness of the pains and troubles to
which you may probably be exposed. As each presents itself ask
yourself: Is there anything intolerable and insufferable in this? You
will be ashamed to own it. And then recollect that it is neither the
past nor the future that can oppress you, but always the present
only. And the ills of the present will be much diminished if you
restrict it within its own proper bounds, and take your soul to task
if it cannot bear up even against this one thing.

37. Does Panthea or Pergamus now sit mourning at the tomb of Verus, or
Chabrias or Diotimus at the tomb of Hadrian? Absurd! And if they were
still mourning could their masters be sensible of it? Or if they were
sensible of it, would it give them any pleasure? Or if they were
pleased with it, could the mourners live for ever? Was it not fate
that they should grow old men and women, and then die? What, then,
would become of the illustrious dead when these faithful souls were
gone? And all this toil for a vile body, naught but blood and
corruption!

38. If you have keen sight, says the philosopher, use it in discretion
and in wisdom.

39. In the constitution of the rational being I discern no virtue made
to restrain justice; but I see continence made to restrain sensual
pleasure.

40. Take away your opinion about the things that seem to give you
pain, and you stand yourself upon the surest ground. What is that
self?--It is reason.--I am not reason, you say.--So be it; then
let not reason pain itself, but leave any part of you which suffers to
its own opinions of the pain.

41. Obstruction of any sense is an evil for the animal nature; so is
the obstruction of any of its impulses. There are other kinds of
obstruction which are evil for the nature of plants. For the rational
nature in like manner, therefore, obstruction of the understanding is
evil. Apply all this to yourself. Do pain and pleasure affect you? Let
the senses look to it. Does anything hinder your designs? If you have
designed without the proper reservations, that in itself is an evil
for you as a reasoning being. If you designed under the general
reservation, you are neither hurt nor hindered. No man can hinder the
proper work of the mind. Nor fire, nor sword, nor tyrant, nor calumny
can reach it, nor any other thing, when it is become even as a sphere,
complete and perfect within itself.

42. I have no right to vex myself who never yet willingly vexed any
one.

43. Each man has his own pleasure. Mine lies in having my ruling part
sound; without aversion to any man, or to any hap that may befall
mankind. Yet let me look on all things with kindly eyes. Let me accept
and use them all according to their worth.

44. See that you secure the benefit of the present time. They who
pursue a fame which is to live after them reflect not that posterity
will be men even as are those who vex them now, and that they too will
be mortal. And afterwards, what shall signify to you the clatter of
their voices, or the opinions they shall entertain about you?

45. Take me up and cast me where you will; I shall have my own
divinity within me serene, that is, satisfied while its every state
and action is according to the law of its proper constitution.

Is any event of such account that my soul should suffer for it or be
the worse; that my soul should become abject and prostrate as a mean
suppliant, or should be affrighted? Shall you find anything that is
worth all this?

46. Nothing can befall a man which is not human fortune. Nothing can
happen to an ox, to a vine, or to a stone which is not the natural
destiny of their species. If, then, that alone can befall anything
which is usual and natural, what cause is there for indignation?
Universal Nature hath brought nothing upon you which you cannot bear.

47. When you are grieved about anything external it is not the thing
itself which afflicts you, but your judgment about it. This judgment
it is in your power to efface. If you are grieved about anything in
your own disposition, who can prevent you from correcting your
principles of life? If you are grieved because you do not set about
some work which seems to you sound and virtuous, go about it
effectually rather than grieve that it is undone.--But some superior
force withstands.--Then grieve not, for the fault of the omission
lies not in you.--But life is not worth living with this undone.--
Quit life then, in the same kindly spirit as though you had done it,
and with goodwill even to those who withstand you.

48. Remember that the governing part becomes invincible when,
collected into itself, it is satisfied in refusing to do what it would
not, even when its resistance is unreasonable. What then will it be
when, after due deliberation it has fixed its judgment according to
reason? The soul, thus free from passions, is a strong fort; nor can a
man find any stronger to which he can fly and become henceforth
invincible. The man who has not discerned this is ignorant. He who has
discerned and flies not thither is miserable.

49. Pronounce no more to yourself than what appearances directly
declare. It is told you that so-and-so has spoken ill of you. This
alone is told you, and not that you are hurt by it. I see my child is
sick; this only I see. I do not see that he is in danger. Dwell thus
upon first appearances; add nothing to them from within, and no harm
befalls you: or rather add the recognition that all is part of the
world's lot.

50. Is the gourd bitter? Put it from you. Are there thorns in the way?
Walk aside. That is enough. Do not add, ``Why were such things brought
into the world?'' The naturalist would laugh at you, just as would a
carpenter or a shoemaker, if you began fault-finding because you saw
shavings and parings from their work strewn about the workshop. These
craftsmen have places where they can throw away this rubbish, but
universal Nature has no such place outside her sphere. Yet the wonder
of her art is that, having confined herself within certain bounds, she
transforms into herself all things within her scope which seem to be
corrupting, or waxing old and useless; and out of them she makes other
new forms; so that she neither needs matter from without nor a place
where to cast out her refuse. She is satisfied with her own space, her
own material, and her own art.

51. Be not languid in action, nor confused in conversation, nor vague
in your opinions. Let there be no sudden contractions or
forth-sallyings of your soul. In your life be not over-hurried.

Men slay you, cut you to pieces, pursue you with curses. What has this
to do with your soul remaining pure, prudent, temperate, and just?
What if some one, standing by a clear sweet fountain, should reproach
it? It would not cease to send forth its refreshing waters. Should he
throw into it mud or dung, it will speedily scatter them and wash them
away, and be in nowise stained thereby. How then shall you get this
perpetual living fount within you? If you reserve yourself unto
liberty every hour you live, in a spirit of calmness, simplicity, and
modesty.

52. He who knows not what the Universe is knows not what is his place
therein. He who knows not for what end it was created, knows not
himself and knows not the world. He who is deficient in either of
these parts of knowledge cannot even say for what end he himself was
created. What sort of man then does he appear to you who pursues the
applause or dreads the anger of those who know neither where nor what
they are?

53. Do you wish to be praised by a man who curses himself thrice
within an hour? Can you desire to please one who is not pleased with
himself?  Can he be pleased with himself who repents of almost
everything he does?

54. No longer be content to breathe in harmony with the air which
surrounds you; but set about feeling in sympathy with the intelligence
which embraces all things. For the power of that intelligence is no
less diffused, and no less pervasive for all who can draw it in, than
is the virtue of the air for him who can breathe it.

55. There is no universal wickedness to hurt the world; and the
particular wickedness of any individual hurts not another. It hurts
himself alone, and even he has this gracious privilege that, as soon
as he desires it, he may be free from it altogether.

56. To my will the will of another is as indifferent as his poor
breath and flesh. And how much soever we were formed for the sake of
each other, yet the governing part of each of us has its own proper
power; otherwise the vice of another might become my own misery. God
thought fit that this should not be; lest it should be in another's
power to make me unhappy.

57. The sun seems to us diffused everywhere, pervasive of all things,
yet never exhausted. This diffusion is a sort of extension, and hence
the Greek word for rays is thought to be derived. You may observe the
nature of a ray if you see it entering through some small hole into a
darkened chamber. Its direction is straight; and it is reflected
around when it falls upon any solid body, which shuts it off from the
air beyond. There it stands and does not slip or fall. Now, such
should be the flow and diffusion of the understanding; never
exhausted, always extending; not violently or furiously dashing
against the obstacles that meet it, nor falling aside, but resting
there and illuminating whatever will receive it. That which will not
transmit the light does but deprive itself of radiance.

58. He who dreads death dreads either the extinction of all sense or
the experience of a new one. If all sense be extinguished, there can
be no sense of evil. If a different sort of sense be acquired you
become a different creature, and do not cease to live.

59. Men were created the one for the other. Teach them better then, or
bear with them.

60. Mind moves in one way, and an arrow in another. The mind, when
cautiously proceeding, or when casting round in deliberation about
what to pursue, is nevertheless carried onward straight toward its
proper mark.

61. Penetrate into the governing part of others; and also allow others
to enter into your own.

\terminus{}
\chapter[He who does injustice...]{}

1. \textsc{He who does injustice} commits impiety. For since universal Nature
has formed the rational animals for one another; each to be useful to
the other according to his merit, and never hurtful; he who
transgresses this her will is clearly guilty of impiety against the
most ancient and venerable of the Gods.

He who lies sins against the same divinity. For the nature of the
whole is the nature of all things which exist; and things which exist
are akin to all that has come to be. Nature, indeed, is called truth,
and is the first cause of all truths. He, then, that lies willingly is
guilty of impiety, in so far as by deceiving he works injury: and he
also who lies unwillingly, in so far as he is out of tune with
universal Nature, and in so far as he works disorder in the Universe
by fighting against its design. He is at war with Nature who sets
himself against the truth. He has neglected the means with which
Nature furnished him, and cannot now distinguish false from true.

He, too, who pursues pleasure as good, and shuns pain as evil, is
guilty of impiety. Such a one must needs frequently blame the common
nature for unseemly awards of fortune to bad and to good men. For the
bad often enjoy pleasures and possess the means to attain them, and
the good often meet with pain and with what causes pain. Again, he who
dreads pain must sometimes dread a thing which will make part of the
world order, and this is impious. And he who pursues pleasure will not
abstain from injustice, and this is clear impiety. In those things to
which the common nature is indifferent (for she had not made both,
were she not indifferent to either), he who would follow Nature ought,
in this also, to be of like mind with her, and shew the like
indifference. And whoever is not indifferent to pain and pleasure,
life and death, glory and ignominy, all of which universal Nature uses
indifferently, is clearly impious. By Nature using them indifferently,
I mean that they befall indifferently all beings which exist, and
ensue upon others in the great chain of consequence which began in the
primal impulse of Providence. Providence, in pursuance of this
impulse, and starting from a definite beginning, set about this fair
structure of the universe when she had conceived the plan of all that
was to be, and appointed the distinct powers which were to produce the
several substances, changes, and successions.

2. It were the more desirable lot to depart from among men,
unacquainted with falsehood, hypocrisy, luxury, or vanity. The next
choice were to expire when cloyed with these vices. Have you then
chosen rather to abide in evil; or has experience not yet persuaded
you to fly from amidst the plague? For a corruption of the mind is far
more a plague than any pestilential distemper or change in the
surrounding air we breathe. The one is pestilence to animals as
animals: but the other to men as men.

3. Despise not death; but receive it well content, as one of the
things which Nature wills. For even as it is to be young, to be old,
to grow up, to be full grown; even as it is to breed teeth, and beard,
and to grow grey, to beget, to go with child, to be delivered; and to
undergo all the effects of nature which life's seasons bring; such is
it also to be dissolved in death. It becomes not therefore a man of
wisdom to be careless, or impatient, or ostentatiously contemptuous
about death; he should rather await its coming as one of the
operations of nature. Even as now you await the season when the child
of your wife's body shall issue into the light, await the hour when
your soul shall fall out of these its teguments. If you wish for the
common sort of comfort, here is a thought which goes to the heart. You
will be completely resigned to death if you consider the things you
are about to leave, and the morals of that confused crowd from which
your soul is to be disengaged. It is far from right to be offended
with them. It is even your duty to have a tender care for them, and to
bear with them mildly. Yet remember that the parting, when it comes,
will not be with men who think as you think. For the only thing which,
if it might be, could hold you back and detain you in life, would be
to live with those who had reached the same principles of life as
you. But, as it is, you, seeing how great is the fatigue and toil
arising from the jarring courses of those who live together, may cry:
``Haste, death!  lest I, too, should forget myself''.

4. The sinner sins against himself. The wrong-doer wrongs himself by
making himself evil.

5. Men are often unjust by omissions as well as by actions.

6. Be satisfied with your present opinion, if certain; with your
present course of action, if social; with your present mood, if well
pleased with all that comes upon you from without.

7. Wipe out impression; stay impulse; quench desire; and keep the
governing part master of itself.

8. The soul distributed among the irrational animals is one. Rational
beings, on the other hand, partake of one reasoning intelligence. Even
so, there is one earth to all things earthy; and, for all of us who
are endowed with sight and breath, there is one light by which to see,
one air to breathe.

9. All things that share a common quality are strongly drawn to that
which is of their own kind. The earthy tends towards the earth; fluids
flow together, aerial bodies likewise; and naught but force prevents
their confluence. Fire rises upward on account of the elemental fire;
and it is so ready to join in kindling with all the fire that is here
that any matter pretty dry is easily set on fire, because that which
hinders its kindling is the weaker element in its composition. Thus
also, then, whatever partakes of the common intellectual nature
hastens in like manner, or even more markedly, towards that which is
akin to it. For the more it excels other natures, the stronger is its
tendency to mix with and adhere to its kind. Accordingly, among
irrational creatures we find swarms of bees, herds of cattle, nurture
of the young, and love, of a sort. For even in animals there is a
soul; and in the more noble natures a mutual attraction is found to be
at work, such as does not exist in plants, or stones, or wood. Among
the rational animals, again, there are societies and friendships,
families and assemblies; and, in war, treaties and truces. Among
beings still more excellent, there subsists, though they be placed far
asunder, a certain kind of union, as among the stars. Thus ascent in
the scale can produce a sympathy even in things that are widely
distant. But mark what happens among us. It is only intellectual
beings who forget the social concern for one another, and the mutual
tendency to union. Here alone the social confluence is not seen. Yet
are they environed and held by it, though they strive to escape; and
nature always prevails. Observe and you will see my meaning: for
sooner may one find some earthy thing which joins with nothing earthy,
than a man severed and separate from all men.

10. Man, God, and the Universe, all bear fruit; and each in their own
season. Custom indeed has appropriated the expression to vines and the
like; but that is nothing. Reason has its fruit both for all men and
for itself, and produces just such other things as reason itself is.

11. If you can, teach men better. If not, remember that the virtue of
charity was given you to be used in such a case. Nay, the Gods are
patient with them, and even aid them in their pursuit of some things
such as health, wealth, and glory, so gracious are they! You may be so
too. Who hinders you?

12. Bear toil and pain, not as if wretched under it, nor as courting
pity or admiration. Wish for one thing only; always to act or to
refrain as social wisdom requires.

13. To-day I have escaped from all trouble; or rather I have cast out
all trouble from me. For it was not without but within, in my own
opinions.

14. All things are, in our experience, common; in their continuance
but for a day; and in their matter sordid. All things now are as they
were in the times of those we have buried.

\newpage

15. Things stand without, by themselves, neither knowing or declaring
aught to us concerning themselves. What is it then that pronounces
upon them? The ruling part.

16. It is not in passive feeling, but in action, that the good and
evil of the rational animal formed for society consists. Similarly his
virtue or his vice lies not in feeling but in action.

17. To the stone thrown up it is no evil to fall; no good to rise.

18. Penetrate the souls of men, and you will see what judges you fear,
and how they sit in judgment on themselves.

19. All things are in change. You yourself are under continual
transmutation, and, in some sort, corruption. So is the whole
universe.

20. Another's sin you must leave with himself.

21. The ceasing of any action, the extinction of any keen desire, or
of any opinion, is as it were a death to them. This is no evil. Think
again of the ages of your life; childhood, youth, manhood, old
age. Each change of these was a death. Is there anything to dread
here? Think now of your life as it was, first under your grandfather,
then under your mother, then under your father; and, as you find there
many other alterations, changes, and endings, ask yourself: Is there
anything to dread here? Thus neither is there anything to dread in the
cessation, ending, and change of your whole life.

22. Make swift appeal to your own ruling part, to that of the
Universe, and to his who has offended you. To your own, that you may
make it a mind disposed to justice; to that of the Universe, that you
may remember of what you are a part; and to his, that you may know
whether he has acted in ignorance or by design, and that you may also
reflect that he is your kinsman.

23. You yourself are a part of a social system necessary to complete
the whole. Accordingly, let your every action be a similar part of the
social life. And if any action has not its reference, either immediate
or distant, to the common good as its end, this action disorders your
life and frustrates its unity. It is sedition like that of the man
who, in a commonwealth, does all in his power to sever himself from
the general harmony and concord.

24. Children's quarrels! Child's play! Poor spirits carrying about
dead corpses! Such is our life. The `Masque of the Dead' is
intelligible by comparison.

25. Go to the quality of the cause; abstract it from the material, and
contemplate it by itself. Determine then the time: how long, at
furthest, this thing, of this peculiar quality, can naturally subsist.

26. You have endured innumerable sufferings by not being satisfied
with your own ruling part when it does the things which it was formed
to do. Enough then of that.

27. When another reproaches or hates you, or utters anything to that
purpose; go to his soul; enter in there; and look what manner of man
he is. You will see that you need not trouble yourself to make him
think well or ill of you. Yet you should be kindly towards such men,
for they are by nature your friends: and the Gods, too, aid them in
all ways; by dreams, by oracles, and even in the things about which
they are most eager.

28. The course of things in the world is ever the same; a continual
rotation; up and down, from age to age. Either the Universal Mind
exerts itself in every particular event, in which case you must accept
what comes immediately from it: or it has exerted itself once and for
all, and, as a result, all things go on for ever, in a necessary chain
of consequence: or again atoms and indivisible particles are the
origin of all things. In fine, if there be a God, all is well; and if
there be only chance, you at least need not act by chance.

The earth will presently cover us all; and then this earth will itself
be changed into other forms, and these again into others, and so on
without end. And, if any one considers how swiftly those changes and
transmutations roll on, like one wave upon another, he will despise
all things mortal.

29. The universal cause is like a winter torrent. It sweeps all along
with it. How very little worth are those poor creatures who pretend to
understand affairs of state, and imagine they unite in themselves the
statesman and the philosopher! The frothy fools! Do you, O man! that
which Nature now requires of you. Set about it if you have the means;
and look not around you to see if any be taking notice, neither hope
to realize Plato's Republic. Be satisfied if the smallest thing go
well. Consider even such an event as no small matter. For who can
change the opinions of men? And without change of opinion what is
their state but a slavery, under which they groan, while they pretend
to obey? Come now; speak of Alexander, Philip, and Demetrius of
Phalerum. They know best whether they understood what the common
nature required of them, and whether they trained themselves
accordingly. But, if they designed only to play the tragic hero, no
one has condemned me to do the like. The work of philosophy is simple
and modest. Lead me not astray in pursuit of a vainglorious
stateliness.

30. Look down, as from some eminence, upon the innumerable herds, the
countless solemn festivals, the voyaging of every sort, in tempests
and in calms; the different states of those who come into life, enter
upon life's associations, and leave it in the end. Consider, too, the
life which others have lived formerly, the life they will live after
you, and the life that barbarous peoples are now living. How many of
these know not even your name; how many will quickly forget it; how
many are there who perhaps praise you now, but will shortly blame
you. Reflect, then, that neither is surviving fame a thing of value;
nor present glory; nor anything at all.

31. Let nothing due to a cause outside yourself disturb your calm. In
the workings of the active principle within you let there be justice:
that is a bent of will and a course of action which have social good
as their one end, and so are suited to your nature.

32. You can suppress many of the superfluous troubles which beset you,
for they lie wholly in your own opinion. By this you will give ample
room and ease to your life. You may compass this end by comprehending
the whole Universe in your judgment; by contemplating eternity; and by
reflecting on the swift changes of individual things, thinking how
short is the time from their birth to their dissolution, how immense
the space of ages before that birth, how equally infinite the eternity
which shall succeed that dissolution.

33. All things that you see will quickly perish; and those who behold
them perishing are very soon themselves to die. And he who dies oldest
will be in like case with him who dies before his time.

34. What manner of souls have these men? What is the end of their
striving; and on what accounts do they love and honour? Imagine their
souls naked before you. When they fancy that their censures hurt, or
their praises profit us, how great is their self-conceit!

35. Loss is naught but change; in change is the joy of universal
Nature, and by her all things are ordered well. From the beginning of
ages they have been shaped alike, and to all eternity they will be the
same. How then can you say that all things have been, and ever will be
evil; that among so many Gods there has been found no power to
rectify; but that the Universe is condemned to endure the burden of
never-ending ill?

36. How corrupt is the material substance of every thing, water, dust,
bones, and foulness! Again; marble is but the concrete humour of the
earth, gold and silver its heavy dregs. Our garments are but hair, the
purple dye blood. All else is of a like nature. Breath, too, is just
the same, ever changing from this to that.

37. Enough of this wretched life: enough of repining and apish
trifling. Why are you disturbed? Are any of these troubles new? What
excites you so? Is it the cause?

Then view it well. Is it the matter? View it also well. Besides these
there is nothing. Wherefore at last act with more simplicity and
goodness towards the Gods. Whether you look on this spectacle for a
hundred years or for three it is the same.

38. If he has done wrong, the evil is with him: and perhaps, too, he
has not done wrong.

39. Either all things proceed from one source of intelligence and come
together in one body, in which case the part must not complain of what
comes about for the benefit of the whole; or all is atoms, and there
is nothing else but confused mixture and dissipation. Why then are you
disturbed? Say to your soul: ``Thou art dead: thou art rotten: thou
hast turned beast, joined the herd, and dost feed along with them''.

40. Either the Gods have power or they have none. If they have no
power, why do you pray? If they have power, why do you not choose to
pray to them for power neither to fear, nor to desire, nor to be
grieved over any of these external things, rather than for their
presence or their absence? Surely, if the Gods can aid man at all,
they can aid him in this. But perhaps you will say ``the Gods have put
this in my own power.'' Then is it not better to use that which is in
your own power and preserve your liberty, than to set your heart on
what is beyond your power and become an abject slave? And who has told
you that the Gods aid us not in these things also which are in our
power? Begin to pray about them and you will see. One man prays: ``May
I possess that woman!'' Do you pray: ``May I have no wish to possess
her!'' Another prays: ``May I be delivered from so and so!'' Pray you:
``May I not need to be delivered from him!'' A third cries: ``May I not
lose my child!''  Let your prayer be: ``May I not fear to lose him!'' In
fine, turn your prayers this way, and observe what comes of it.

41. Epicurus says: ``In my sickness my conversations were not about the
diseases of this poor body; nor did I speak of any such things to
those who came to me. I continued to discourse as before on the
principles of natural Philosophy, and was chiefly intent on the
problem of how the mind, though it partakes in the violent commotions
of the flesh, might remain undisturbed and keep guard on its own
proper excellence. I permitted not the physicians,'' he continues, ``to
magnify their office, and vaunt themselves as if they were
doing-something of great moment, but my life continued pleasant and
happy.'' What he did then, in sickness, do you also if ye fall ill, or
suffer any other misfortune. Never to depart from your philosophy
whatever befalls you, never to join in the folly of the vulgar and the
ignorant, is a maxim common to all the schools. Give your mind only to
the business now in hand and to the means whereby it is to be
accomplished.

42. When you are offended by the shamelessness of any man, straightway
ask yourself: Can the world exist without shameless men? It
cannot. Therefore do not demand what is impossible. Your enemy also is
one of these shameless people who must needs be in the universe. Have
the same question also at hand when you are shocked at craft, or
perfidy, or any other sin. For while you remember that it is
impossible that the class should not exist, you will be more
charitable to each particular individual. It is useful also to have
this reflection ready: What virtue has nature given to man wherewith
to combat this fault? Against unreason she has given meekness as an
antidote; against another weakness another power. You are also at full
liberty to set right one who has wandered; now every wrong-doer is
missing his proper aim and has gone astray. And then, in what are you
injured? You will find that none of those at whom you are exasperated
have done anything whereby your intellectual part was like to be the
worse. Now anything which can really harm or hurt you has its
subsistence there, and there alone. And wherein is it strange or evil
that the man untaught acts after his kind? Look if you ought not
rather to blame yourself for not having laid your account with his
being guilty of such faults. Your reason gave you the means to
conclude that it was probable that he would do this wrong; you forgot,
and yet wonder that he has done it. But above all, when you are
blaming any one for faithlessness or ingratitude, turn to
yourself. The fault lies manifestly with you, if you trusted that a
man of such a disposition could keep faith; or if, when you granted
the favour, you did not grant it without ulterior views, and on the
principle that the complete and immediate reward of your action lay in
the doing of it. What would you more, when you have done a man a
kindness? Is it not enough for you that you have acted in this
according to your nature? Do you ask a reward for it? It is as if the
eye were to ask a reward for seeing, or the feet for walking. For just
as these parts are formed for a certain purpose, which when they
fulfil according to their proper structure, they attain their proper
end; so man, formed by nature to do kindness to his fellows, whenever
he acts kindly, or in any other way works for the common good, has
fulfilled the purpose of his creation, and has possession of what is
his own.

\terminus{}
\chapter[Wilt thou ever,...]{}

1. \textsc{Wilt thou ever,} O my soul, be good and single, and one, and naked,
more open to view than the body which surrounds thee? Wilt thou ever
taste of the loving and satisfied temper? Wilt thou ever be full and
without wants, setting thy heart on nothing, animate or inanimate, for
the enjoyment of pleasure; not desiring time for longer enjoyment; nor
place, nor country, nor fine climate, nor congenial company? Wilt thou
be satisfied with thy present state, and well pleased with every
present circumstance? Wilt thou persuade thyself that all things are
thine; that all is well with thee; that all comes to thee from the
Gods; and that what is best for thee is what they are pleased to give,
now and henceforth, for the preservation of that perfected being,
which is good, just, and beautiful; which generates, combines,
embraces, and includes all fleeting things that dissolve to bring
forth others like themselves? Wilt thou never be able to live a fellow
citizen with Gods and men, approving them and by them approved?

2. In so far as you are governed by nature only, observe carefully
what nature demands; then do that freely, if thereby your nature as a
living being be not made worse. Next you must consider what the nature
of a living being demands, and allow yourself everything of this kind
by which your nature as a rational being is not made worse. Now it is
plain that what is rational is also social. Therefore follow these
rules and trouble no further.

3. Whatever happens, Nature has either formed you able to bear it or
unable. If able, then bear it as Nature has made you able, and fret
not. If unable, yet do not fret, for when the trial has consumed you
it too will pass away. Remember, however, that Nature has made you
able to bear whatever it is in the power of your own opinion to make
endurable or tolerable, if only you conceive it profitable or fit to
be borne.

4. If a man is going wrong, instruct him kindly, and shew him his
mistake. If you are unable to do this, blame yourself or none.

5. Whatever happens to you was prearranged for you from all eternity;
and the concatenation of causes had from eternity interwoven your
existence with this contingency.

6. Whether all be atoms, or there be a universal Law of Nature, let it
be laid down first that I am a part of the whole which is governed by
Nature; secondly, that I am associated with other parts like
myself. Mindful of this, since I am a part, I shall not be
dissatisfied with anything appointed me by the whole. For nothing is
hurtful to the part which is profitable to the whole, since the whole
contains nothing unprofitable to itself. All natural systems have this
law in common, and the system of the Universe has another law besides;
namely that it cannot be forced by any external cause to produce
anything hurtful to itself. If therefore I remember that I am part of
such a whole, I shall be satisfied with all that flows therefrom. And,
inasmuch as I am associated with parts like myself, I will do nothing
unsocial; but rather draw to my kind, turn my every endeavour to the
public good, and shun the contrary. In such a course my life must
needs run well, just as you would hold that the life of a citizen runs
well when he passes on from one public-spirited action to another, and
throws himself heartily into every task appointed him by the State.

7. The parts of the whole, I mean the parts which are contained in the
Universe, must necessarily perish; ``perish,'' let us say, meaning
change. Now, if it be a necessary evil for the parts to perish, it
could not be well for the whole that its parts should tend to change
and be constructed to perish in various ways. Did Nature then set out
to injure her own constituent parts, making them so that they are
liable to evil and of necessity fall into it; or did it escape her
notice that this comes to pass? Both suppositions are incredible. And
if, dropping the notion of Nature, one were merely to put it that
things are constituted so, then how ridiculous at the same time to say
that the parts of the Universe are constituted so as to change, and
also to wonder and fret at change or dissolution, as if it were
something against the course of Nature; especially as everything is
dissolved into the elements out of which it arose. For there is either
a scattering of the elements of which a thing was constructed, or a
conversion of these, of the solid into earth, of the spiritual into
air. So that these constituents are resumed into the system of the
Universe, which either undergoes periodical conflagration, or is
renewed by never-ending changes. And do not imagine that you had all
your earthy and aerial matter from your birth. For the whole of this
was an accession of yesterday or the day before, from your food and
from the air you breathed. It is this accession which changes, and not
what your mother bore. And granting that this recent accession may
incline you more to what is individual in your constitution; yet, I
think, it alters nothing of what has just been said.

8. Having taken to yourself these titles: good, modest, true, prudent,
even-tempered and magnanimous, look to it that you change them not;
and, if you should come to lose them, seek them straightway again. And
remember that prudence means for you reasoned observation of all
things, and careful attention; even temper, cheerful acceptance of the
lot appointed by universal Nature; magnanimity, the exaltation of the
thinking part above any pleasant or painful commotions of the flesh,
above vain-glory, above death and all such things. If you steadfastly
maintain yourself in these titles, with no hankering after hearing
them given to you by others, you will be a new man, and a new life
will open for you. For to continue as you have been till now, in the
same life of distraction and defilement, would mark you as a man
devoid of sense, who clings to life like the half-eaten
beast-fighters, who, though covered with wounds and gore, do yet
appeal to be reserved until tomorrow, to be cast again in their
wretchedness to the claws and fangs that lacerated them before. Take
your stand then on these few titles; and if you are able to abide in
them, abide, as one removed to the Islands of the Blest. But if you
perceive that you are falling away, and cannot prevail; have the
courage to retire into some corner where you may hope to prevail, or
else depart from life altogether, not in anger but in all simplicity,
freedom, and modesty, having done at least one thing in life well, by
so leaving it. Now it will greatly help you to be mindful of your
titles, if you recollect that the Gods desire not adulation, but that
reasoning beings should grow in likeness to themselves; and further
that a fig tree is set to bear figs, a dog to hunt, a bee to gather
honey, and a man to do a man's work.

9. Mimes, war, panic, sloth, servility, will wipe out the sacred
maxims which you have gathered by observing Nature and stored in your
mind. You ought to look and act in every case so that not only shall
the task before you be accomplished, but also your theoretic faculty
exercised, and the self-confidence which springs from special
knowledge preserved without ostentation or affected concealment. Will
you ever attain to simplicity; to dignity; to a perfect discrimination
in every case as to what a thing really is, what its true place in the
Universe, what the time it may endure, what its composition, to whom
it may belong, and who can give and take it away?

10. The spider exults when he has captured a fly; one man because he
has taken a little hare, another because he has netted an anchovy,
another because he has hunted down a wild boar or a bear; and another
because he has conquered the Sarmatians. But are they not brigands
all, if you look to their principles.

11. Acquire a method of perceiving how all things change into one
another. Pursue this branch of Philosophy and continually exercise
yourself therein. There is nothing so proper as this for cultivating
greatness of mind. He who does so has already put off the body; and,
having realized how soon he must depart from among men and leave all
earthly things behind him, he resigns himself entirely to justice in
all his own actions, and to the law of the Universe in everything else
which happens. As for what any one may say or think of him or do
against him, he gives it not a thought, but contents himself with
these two things: to do justly what he has in hand, and to love the
lot appointed for him. Such a man has thrown off all hurry and bustle;
and has no other will but this, to keep the straight path according to
the law, and to follow God whose path is ever straight.

12. What need for suspicion when it is open for you to consider what
ought to be done? If you see your way, proceed in it calmly,
inflexibly. If you do not see it, pause and consult the best
advisers. If any other obstacle arise, proceed with prudent caution,
according to the means you have; keeping always close to what appears
just. That is the best to which you can attain: and failure in that is
the only proper miscarriage. He who in everything follows reason is
always at leisure, yet ever ready for action, always cheerful, yet
composed.

13. As soon as you awake ask yourself: Will it be of consequence to
you if what is just and good be done by some other man? It will
not. Have you forgotten what manner of men in bed and at table are
those who make such display in praise and blame of others; what they
do, what they shun and what they pursue; how they steal and how they
rob, not with hands and feet but with their most precious part,
whereby, if a man will, he may gain faith, modesty, truth, law, a good
directing spirit?

14. To Nature, which gives and again resumes all things, the
well-instructed, modest man will say: ``Give what thou wilt; take again
what thou wilt.'' And this he says, not with ostentation, but out of
pure obedience and good will to Nature.

15. What remains to you of this life is little. Live as on a
mountain. For it makes no difference whether we live here or there,
provided we live like citizens everywhere in the world. Let men see
and know you as a man indeed, living according to Nature. If they
cannot endure you, let them slay you. It is better so than to live as
they live.

16. Discourse no more of what a good man should be; but be one.

17. Constantly imagine all time and all existence; and think that
every individual thing is in substance a fig seed, and in time the
turn of an auger.

18. Consider each of the things around you as already dissolving, in a
state of change, and as it were corrupting and being dissipated, or
as, one and all, formed by Nature to die.

19. What sort of men are they when they are eating, sleeping,
procreating, easing nature, and the like? Then see them lording it
over their fellows, puffed up with pride, angry, or issuing judgments
from on high! To how many were they slaves but lately, and why! And in
what case will they shortly be?

20. That is for the advantage of every man which is brought by
universal Nature; and for his advantage at the very time at which she
brings it.

21. ``Earth loves the rain;'' ``and the majestic Ether loves.'' The
Universe loves to bring about whatever is coming to be. I then will
say to the Universe: ``What thou lovest I love.'' Is it not a common
saying that, ``so-and-so loves to happen''?

22. Either you are living here your accustomed life; or you are going
abroad, and that at your own will: or you are dying, and your public
office is discharged. Now, besides these there is nothing. Be
therefore of good courage.

23. Keep this ever clear before you: that a country retreat is just
like any other place. All things here go the same as on a mountain
top, or on the sea beach, or where you will. You may always find that
life of the wise man who, in Platonic phrase, ``makes the city wall
serve him for a shepherd's fold on the mountains''.

24. What is my soul to me? What am I making of it, and to what purpose
am I now using it? Is it void of understanding? Is it loosened and
rent from the great community? Is it glued to, and mingled with, the
flesh so as to follow each fleshly motion?

25. Whoever flies from his master is a runaway. Our master is the law,
and the law-breaker is a runaway; and so is he also who through grief,
or anger, or fear will not acquiesce in something that has happened,
is happening, or will happen, in the course of things predestined by
the all-ruling power which is the law, laying down for every man what
is proper for him. He then who is afraid or grieved or angry, is a
runaway.

26. He who has cast seed into the womb departs; another cause takes
and works upon it and completes the child. How wonderful the result
from such a beginning! The child, again, takes food down its throat;
another cause takes and transforms it into sensation, motion, in a
word into life and strength and other things, how many and surprising!
Consider then these things happening in such hidden ways, and view the
power which produces them just as we perceive the gravitation and
levitation of bodies; not indeed with our eyes, yet none the less
clearly.

27. Continually reflect that all that is happening now happened
exactly in the same way before; and reflect that the like will happen
again. Place before your eyes all that you have ever known from your
own experience or from ancient history; dramas and scenes, all
similar; such as the whole court of Hadrianus, the whole court of
Antoninus, the whole court of Philip, of Alexander, of Croesus. All
these were similar, only the actors different.

28. Imagine every one who is grieved or storms about anything
whatever, to be like the pig in a sacrifice, which kicks and screams
under the knife. Such, too, is he who, on his couch, deplores in
silence, by himself, that we are all tied to our fate. Reflect also
that only to a rational being is it given to submit to what happens
willingly; the bare submission is a necessity upon all.

29. Look attentively on each particular thing you do, and ask yourself
if death be a terror because it deprives you of this.

30. When you are offended at any one's fault, turn at once to yourself
and consider of what similar fault you yourself are guilty; such as
esteeming for good things, money, pleasure, a little glory, or the
like. By fixing your attention on this you will speedily forget your
anger, especially if it occur to you that he acts under compulsion and
cannot do otherwise; else, if it be in your power, relieve him from
the compulsion.

31. When you have seen Satyrio the Socratic, think of Eutyches or
Hymen; when you have seen Euphrates, think of Eutychio or
Silvanus. When Alciphron comes before you, think of Tropaeophorus; and
when Xenophon think of Crito or Severus. When you look upon yourself
think of any of the Caesars, and with every man likewise. Then let
this occur to you: Where, now, are these? Nowhere; or who can tell?
For thus you will see all human things to be smoke and nothingness;
especially if you call to mind that what has once been changed will
never exist again through all the infinity of time. Why then this
concern? And why does it not suffice you to live out your short span
in well ordered wise? What material, what a subject for Philosophy you
are shunning! For what are all earthly things but exercises for the
rational power, when it has viewed all things that occur in life
accurately and in their natural order? Abide then until you have
assimilated all these things, as a strong stomach assimilates every
variety of food, as a bright fire turns whatever you throw upon it
into flame and radiance.

32. Let no man have it in his power to say with truth of you that you
are not a man of simplicity, candour, and goodness. But let him prove
to be mistaken who holds any such opinion of you. This is quite in
your power; for who shall hinder you from being good and
single-hearted?  Only do you determine to live no longer if you cannot
be such a man; for neither does reason require, in that case, that you
should.

33. In the present matter what is the soundest that can be done or
said?  For, whatever that may be, you are at liberty to do or say
it. Make no excuses as if hindered. You will never cease from groaning
until your disposition is such that what luxury is to men of pleasure,
that to you is doing what is suitable to the constitution of man on
every occasion that is thrown or falls in your way. You should regard
as enjoyment everything which you are at liberty to do in accordance
with your own proper nature; and this liberty you have everywhere. Now
to the cylinder it is not given to move everywhere with its proper
motion; nor to water, nor to fire, nor to any other thing that is
governed by a natural law only, or by a soul irrational; for there are
many circumstances which constrain and stop them. But intelligent
reason can pursue through every obstacle the course for which it was
created, and which it wills to follow. Set before your eyes this ease
with which reason makes its way through all obstacles, as fire goes
upwards, a stone downwards, or a cylinder down a slope, and seek for
nothing further. The rest of man's difficulties are merely of the
body, the lifeless part of him; or else they are such as cannot crush,
or in any way injure him save through opinion, or the surrender of
reason itself: otherwise he who suffered by them would himself
straightway become evil. In the case of all other organisms, when
mishap befalls, the sufferer is thereby rendered worse. But in this
respect it may be said that a man becomes better and more praiseworthy
by rightly using his circumstances. In fine, remember that nothing
which hurts not the city hurts the man who is by nature a citizen; nor
does that hurt the city which hurts not the law. Now, none of the
things called misfortunes can hurt the law. Accordingly, what hurts
not the law can hurt neither city nor citizen.

34. To the man who is penetrated with true principles, the shortest, the
most common hint is a sufficient memorial to keep him free of sorrow
and fear. Such as:--

\begin{quote}
      Some leaves the winds blow down: the fruitful wood \\
      Breeds more meanwhile, which in springtide appear. \\
      Of men thus ends one race, while one is born.
\end{quote}

Your children are leaves; leaves, too, the creatures who confidently
cry aloud and deal out eulogy, or, it may be, curses; or who carp and
jeer in secret. Leaves, likewise, are they who transmit our fame to
posterity. All these ``in springtide appear;'' then the wind shakes them
down, and the forest grows more to take their places. Shortness of
life is common to all things, yet you shun and pursue them, as though
they were to have no ending. But a little and you will fall asleep;
and anon others shall mourn for him who carried your bier.

35. The healthy eye ought to look on everything visible, and not to
say, ``I want green,'' like an eye that is diseased. Sound hearing or
sense of smell ought to be ready for all that can be heard or smelt;
and the healthy stomach should be equally disposed for all sorts of
food, as a mill for all that it was built to grind. So also the
healthy mind should be ready for all things that happen. That mind
which says, ``Let my children be spared, and let men applaud my every
action,'' is as an eye which begs for green, or as teeth which require
soft food.

36. There is no man so happily fated but that when he is dying some
bystander will rejoice at the doom which is coming upon him. Were he a
virtuous and wise man; will not some one at the last say within
himself: ``At last I shall breathe freely, unoppressed by this
pedagogue. He was not indeed hard on any of us; but I always felt that
he tacitly condemned us''? This they would say of a good man. But, in
my own case, how many more reasons are there why a multitude would
rejoice to be rid of me? You will reflect on this when dying, and
depart with the less regret when you consider: ``I am leaving a life
from which my very partners, for whom I toiled, and prayed, and
planned, are wishing me to begone; hoping, it may be, to gain some
additional advantage from my departure.'' Why then should one strive
for a longer sojourn here? Yet let not your parting with them be less
pleasant on this account. Preserve your own character, remain to them
friendly, benevolent, gracious. On the other hand, depart from your
fellow-men, not as if torn away; but let your going be like that of
one who dies an easy death, whose soul is gently released from the
body. Nature knit and cemented you to your fellows, but now she parts
you from them. I part, then, as from relations, not reluctant, but
unconstrained. For death, too, is a thing accordant with nature.

37. Accustom yourself as much as possible, when any one takes any
action, to consider only: To what end is he working? But begin at
home; and examine yourself first of all.

38. Remember that the mover of the puppet strings is the hidden
principle within. It is that which is eloquence; that which is life;
that, if I may say so, which is the man. Never, in your imagination,
confound that principle with the surrounding earthen vessel and the
little organs that are kneaded on to it. 
\newpage
Excepting that they grow upon
us, they are like the carpenter's axe; since, without the moving and
restraining principle, none of these parts in itself is of any greater
service than the shuttle to the weaver, the pen to the writer, or the
whip to the charioteer.

\terminus{}
\chapter[These are the characteristics...]{}

1. \textsc{These are the characteristics} of the rational soul: It beholds
itself; it regulates itself in every part; it fashions itself as it
wills; the fruit it bears itself enjoys, whereas the products of
plants and of the lower animals are enjoyed by others. It reaches its
individual end, wheresoever the close of life may overtake it. In a
dance or an actor's part any interruption spoils the completeness of
the whole action. Not so with the rational soul. At whatever point in
its action, or wheresoever it is overtaken by death, it makes its part
complete and all-sufficient; so that it can say, ``I have received what
is mine.'' Also it ranges through the whole universe, and the void
around it, and discerns its plan. It stretches forth into limitless
eternity, and grasps the periodical regeneration of all things, seeing
and comprehending that those who come after us will see nothing new,
and that those that went before saw no more than we have seen. Nay, a
man of forty, of any tolerable understanding, has, because of the
uniformity of things, seen, in a manner, all that has been or will
be. Characteristic of the rational soul also are:--Love to all around
us, truth, modesty; and respect for itself above all other things,
which is characteristic also of the general law. Thus there is no
discordance between right reason and the reason of justice.

2. You will think little of a pleasing song, a dance, or a gymnastic
display, if you analyse the melody into its separate notes, and ask
yourself regarding each, ``Does this impress me?'' You will blush to own
it; and so also if you analyse the dance into its single motions and
postures, and if you similarly treat the gymnastic display. In general
then, except as regards virtue and virtuous action, remember to recur
to the constituent parts of things, and by dissecting to despise them;
and transfer this practice to life as a whole.

3. How happy is the soul that stands ready to part from the body when
it must, and either to be extinguished or to be scattered, or to
survive!  But let this readiness arise from individual judgment, not
from mere obstinacy, as with the Christians, but deliberately, with
dignity, and with no affected air of tragedy; so that others may be
led to a like disposition.

4. Have I done anything for the common good? Is not this itself my
advantage? Let this thought be ever with you, and desist not.

5. What is your art? Well doing. And how else can this come than from
sound general principles regarding Nature as a whole, and the
constitution of man in particular?

6. First of all, tragedy was introduced to remind us that certain
events happen, and are fated to happen as they do; and to teach us
that what entertains us on the stage should not grieve us on the
greater stage of the world. You see that such things must be
accomplished; and that even they bore them who cried aloud, O
Cithaeron! Our dramatic poets have said some excellent things;
especially the following:--

\begin{quote}
    Me and my children, if the Gods neglect, \\
    It is for some good reason--
\end{quote}

and again,

\begin{quote}
    Vain is all anger at external things;
\end{quote}

and,

\begin{quote}
    To reap our life like ears of ripened corn--
\end{quote}

and the like.

And after tragedy came the Old Comedy, using a schoolmaster's freedom
of speech, and employing plain language with great profit to inculcate
the duty of humility. To this end Diogenes used a method much the
same. Next consider the nature of the Middle Comedy; and lastly for
what purpose the New was introduced, which gradually degenerated into
the mere ingenuity of artificial mimicry. It is well known that some
useful things were said by the New Comic Writers; but what useful end
had they in view in all their accumulated poetry and playmaking?

7. How manifest it is that no other course of life was more adapted to
the practice of philosophy than that which now is yours.

8. A branch cut off from its adjacent branch must necessarily be
severed from the whole tree. Even so a man, parted from any
fellow-man, has fallen away from the whole social community. Now a
branch is cut off by some external agency; but a man by his own action
separates himself from his neighbour--by hatred and aversion, unaware
that he has thus torn himself away from the universal polity. Yet
there is always given us the good gift of Zeus, who founded the great
community, whereby it is in our power to be reingrafted on our kind,
and to become once more, natural parts completing the whole. Yet the
frequent happening of such separations, makes the reunion and
restoration of the separated member more and more difficult. And in
general a branch which has grown from the first upon a tree, and
remained a living part of it, is not like one which has been cut and
reingrafted; as the gardeners would say, they are of the same growth
but of different persuasion.

9. As those who oppose you in the path of right reason have no power
to divert you from sane action, so let them not turn you away from
amenity towards themselves. Be watchful alike to persist in stable
judgment and action, and in meekness towards those who would hinder or
otherwise molest you. It is equally weak to grow angry with them or to
desist from action and submit to defeat. Both are equally deserters--
he who runs away, and he who refuses to stand by friend and kinsman.

10. Nature cannot be inferior to Art. The Arts are but imitations of
Nature. If this be so, that Nature which is the most perfect and
comprehensive of all cannot be inferior to the best artistic
skill. Now all Arts use inferior material for higher purposes; so also
then does universal Nature. Hence the origin of justice, from which
again the other virtues spring. Justice cannot be preserved if we are
solicitous about things indifferent, if we are easily deceived, rash,
and changeable.

11. If those things, the pursuit and avoidance of which trouble you,
come not to you; but, as it happens, you go to them; then let your
judgment be at peace concerning them, they will remain motionless, and
you will no more be seen pursuing or avoiding them.

12. The sphere of the soul attains to perfect shape when it neither
expands to what is without, nor contracts upon what is within; neither
wrinkles nor collapses, but shines with a radiance whereby it discerns
the truth of all things, both without itself and within.

13. Does any man contemn me? Let him look to that. And let me look to
it that I be found doing or saying nothing worthy of his
contempt. Does any one hate me? That is his affair. I shall be kind
and good-natured to every one, and ready to shew his mistake to him
that hates me; not in order to upbraid him, or to make a show of my
patience, but from genuine goodness, like Phocion, if he indeed was
sincere. Your inward character should be such that the Gods may see
you neither angry nor repining at anything. What evil is it for you
now to act according to your nature, and to accept now what is
seasonable to the nature of the Universe; you, a man appointed to do
some service for the common good?

14. Although they despise, yet they flatter one another. Although they
desire to overtop, yet they cringe to one another.

15. How rotten and insincere is his profession who says, ``I mean to
deal straightforwardly with you.'' What are you doing, man? There is no
need for such a preface. It will appear of itself. Such a profession
should be written clearly on your forehead. A man's character should
shine forth clearly from his eyes; as the beloved sees that he is so
in the glances of those that love him. The straightforward, good man
should be like one of rank odour who can be recognised by the passer
by as soon as he approaches, whether he will or no. The ostentation of
straightforwardness is the knife under the cloak. Nothing is baser
than wolf-friendship. Shun it above all things. The good,
straightforward, kindly man bears all these qualities in his eyes, and
is not to be mistaken.

16. To live the best life is within the power of the soul, if it be
indifferent to indifferent things. And it will be indifferent if it
looks on all such things, severally and wholly, with discrimination;
mindful that not one of them can impose upon us an opinion concerning
itself, or can come of itself to us. Things stand motionless without;
and it is we that form opinions about them within, and, as it were,
write these opinions upon our hearts. We may avoid so writing them;
or, if one has crept in unawares, we may instantly blot it out. 'Tis
but for a short time that we shall need this vigilance, and then life
will cease. For the rest, why should we hold this to be difficult? If
it be according to Nature, rejoice in it, and it will become easy for
you. If it be contrary to Nature, search out what suits your nature,
and follow it diligently, even though it be attended with no glory;
for every man will be forgiven for seeking his own proper good.

17. Consider whence each thing came, of what it was compounded, into
what it will be changed, how it will be with it when changed, and that
it will suffer no evil.

18. As to those who offend me, let me consider:

First, how I am related to mankind; that we are formed, the one for
the other; and that, in another respect, I was set over them as the
ram over the flock, and the bull over the herd. Consider yet more
deeply, thus:--There is either an empire of atoms, or an intelligent
Nature governing the whole. If the latter, the inferior beings are
created for the superior, and the superior for each other.

Secondly: Consider what manner of men they are at table, in bed, or
elsewhere; and especially by what principles they hold themselves
bound, and with what arrogance they entertain them.

Thirdly: If they act rightly, we ought not to take it amiss; and, if
not rightly, manifestly they do so without intention and in
ignorance. For no soul is willingly deprived of truth, or of the
faculty of treating every man as he deserves. Accordingly men are
grieved to be called unjust, ungrateful, greedy, and, in short,
sinners against their neighbours.

Fourthly: You yourself do often sin, and are no better than
another. And, if you abstain from certain sins, still you have the
disposition to commit them, even if through cowardice, fear for your
character, or other meanness, you hold back.

Fifthly: You cannot even be perfectly sure that wrong has been done,
for many things admit of justification. And, generally speaking, a man
must have learned much before he can pronounce surely upon the conduct
of others.

Sixthly: When you are vexed or worried overmuch, remember that man's
life is but for a moment, and that in a little we shall all be laid to
rest.

Seventhly: It is not the acts of others that disturb us. Their actions
reside in their own souls. Our own opinions alone disturb us. Away
with them then; will that you entertain no thought of calamity
befallen you, and the anger is gone. But how remove them?  By
reasoning that there is no dishonour; for, if you hold not that
dishonour alone is evil, verily you must fall into many crimes, you
may become a robber, or any sort of villain.

Eighthly: How much worse evils we suffer from anger and grief about
certain things than from the things themselves about which these
passions arise.

Ninthly: Meekness is invincible if it be genuine, without simper or
hypocrisy. For what can the most insolent of men do to you, if you
persist in civility towards him; and, if occasion offers, admonish him
gently and deliberately, shew him the better way at the very moment
that he is endeavouring to harm you? ``Nay, my son; we were born for
something better. No hurt can come to me; it is yourself you hurt, my
son.'' And point out to him delicately, and as a general principle, how
the matter stands; that bees and other gregarious animals do not act
like him. But this must be done without irony or reproach, rather with
loving-kindness and no bitterness of spirit; not as though you were
reading him a lesson, or seeking admiration from any bystander, but as
if you designed your remarks for him alone, though others may be
present.

Remember these nine precepts as gifts received from the Muses; and
begin now to be human for the rest of your life. Beware equally of
being angry with men and of flattering them. Both are unsocial and
lead to mischief. In all anger recollect that wrath is not becoming to
a man; but that meekness and gentleness, as they are more human, are
also more manly. Strength and nerves and courage are the portion of
the meek and gentle man; and not of the irascible and impatient. For
the nearer a man attains to freedom from passion, the nearer he comes
to strength. A weak man in grief is like a weak man in anger. Both are
hurt, and both give way.

If you want a tenth gift, from the Leader of the Muses, take this:--
To expect the wicked not to sin is madness. It is to expect an
impossibility. But to allow them to injure others, and to forbid them
to injure you, is foolish and tyrannical.

19. There are four states of the soul against which you must
continually and especially be upon your guard; and which, when
detected, should be effaced, by remarking thus of each. ``This thought
is unnecessary. This tends to social dissolution. You could not say
this from your heart; and to speak otherwise than from the heart you
must regard as the most absurd conduct.'' And, fourthly, whatever
causes self-reproach is an overpowering or subjection of the diviner
part within you to the less honourable and mortal part, the body, and
to its grosser tendencies.

20. The serial and igneous parts of which you are compounded, although
they naturally tend upwards, nevertheless obey the general law of the
Universe, and are retained here in composition. The earthy and humid
parts of you, though they naturally tend downwards, are nevertheless
supported and remain where they are, although not in their natural
situation. Thus the elements, wheresoever placed by the superior
power, obey the whole; waiting till the signal shall sound again for
their dissolution. Is it not grievous that the intellectual part alone
should be disobedient, and fret at its function? Yet is no violence
done to it, nothing imposed contrary to its nature. Still it is
impatient, and tends to opposition. For all its tendencies towards
injustice, debauchery, wrath, sorrows, and fears are so many
departures from Nature. And, when the soul frets at any particular
event, it is deserting its appointed station. It is formed for
holiness and piety toward God, no less than for justice. These last
are branches of social goodness even more venerable than the practice
of justice.

21. He whose aim in life is not always one and the same cannot himself
be one and the same through his whole life. But singleness of aim is
not sufficient, unless you consider also what that aim ought to
be. For, as there is not agreement of opinion regarding all those
things which are reckoned good by the majority, but only as regards
some of them such as are of public utility; so your aim should be
social and political. For he alone who directs all his personal aims
to such an end can reach a uniform course of conduct, and thus be ever
the same man.

22. Remember the country mouse and the town mouse; and how the latter
feared and trembled.

23. Socrates called the maxims of the vulgar hobgoblins, bogies to
frighten children.

24. The Spartans at their public shows set seats for strangers in the
shade, but sat themselves where they found room.

25. Socrates made this excuse for not going to Perdiccas upon his
invitation: ``Lest I should come to the worst of all ends, by receiving
favours which I could not return''.

26. In the writings of the Ephesians there is a precept, frequently to
call to remembrance some of those who cultivated virtue of old.

27. The Pythagoreans recommended that we should look at the heavens in
the morning, to put us in mind of beings that go on doing their proper
work uniformly and continuously; and of their order, purity and naked
simplicity; for there is no veil upon a star.

28. Think of Socrates clad in a skin, when Xanthippe had taken his
cloak and gone out; and what he said to his friends, who were ashamed,
and would have left him when they saw him dressed in such an
extraordinary fashion.

29. In writing and reading you must be led before you can lead. Much
more is this so in life.

30. Yourself a slave, your speech cannot be free.

31. And my heart laughed within me.

32. Virtue herself they blame with harshest words.

33. To look for figs in winter is madness; and so it is to long for a
child that may no longer be yours.

34. Epictetus said that, when you kiss your child, you should whisper
within yourself: ``To-morrow perhaps he may die.'' ``Ill-omened words''!
say you. ``The words have no evil omen,'' says he, ``but simply indicate
an act of Nature. Is it of evil omen to say the corn is reaped''?

35. The green grape, the ripe cluster, the dried grape are all
changes, not into nothing, but into that which is not at present.

36. No man can rob you of your liberty of action; as has been said by
Epictetus.

37. He tells us also that we must find out the true art of assenting;
and in treating of our impulses he says that we must be vigilant in
restraining them, that they may act with proper reservation, with
public spirit, with due sense of proportion; also that we should
refrain utterly from sensual passion; and not be restive in matters
where we have no control.

\newpage

38. The contention is not about any chance matter, said he, but as to
whether we are insane or sane.

39. What do you desire? says Socrates. To have the souls of rational
beings or of irrational? Rational. Rational of what kind, virtuous or
vicious? Virtuous. Why then do you not seek after such souls? Because
we have them already. Why then do you fight and stand at variance?

\terminus{}
\chapter[All that you desire...]{}

1. \textsc{All that you desire} to compass by devious means is yours already,
if you will but freely take it. That is to say, if you will leave
behind you all that is past, commit the future to Providence, and
regulate the present in piety and justice. In piety that you may love
your appointed lot; for Nature gave it to you and you to it. In
justice, that you may speak the truth with-out constraint or guile;
that you may do what is lawful and proper; that you may not be
hindered by the wickedness of others, or by their opinion, or their
talk, or by any sensation of this poor surrounding body, for the part
concerned may look to that. If then, now that you are near your exit,
setting behind you all other things, you will hold alone in reverence
your ruling part, the spirit divine within you; if you will cease to
dread the end of life, but rather fear to miss the beginning of life
according to Nature, you will be a man, worthy of the ordered Universe
that produced you; you will cease to be a stranger in your own
country, gaping in wonder at every daily happening, caught up by this
trifle or by that.

2. God beholds all souls bare and stripped of these corporeal vessels,
husk, and refuse. By his intelligence alone he touches that only which
has been instilled by him and has emanated from himself. If you would
but inure yourself to do the like, you would be eased of many a
torment. For he who regards not the surrounding flesh will not waste
his leisure in thinking about vesture, house, or fame, or other mere
external furniture or accoutrement.

3. Three parts there are of which you are compact; body, soul,
intelligence. Of these the two first are yours in so far as they must
have your care; the third only is properly your own. And if you will
cast away from yourself, that is from your mind, all that others do or
say, all that you yourself have done or said, all your fears for the
future, all the uncontrollable accompaniments of the body that
envelops you and of its congenital soul, and all that is whirled in
the besieging vortex that races without, so that your intellectual
power, made pure, and set above the accidents of fate, may live its
own life in freedom, just, resigned, veracious; if, I repeat, you cast
out from your soul all comes of excessive attachment either to the
past or to the future, then you will become in the words of
Empedocles,

\begin{quote}
    A faultless sphere rejoiced in endless rest.
\end{quote}

You will study to live the only life there is to live, to wit the
present; and you will be able, till death shall come, to spend what
remains of life in noble tranquillity, at peace with the spirit
within.

4. I have wondered often how it comes that, while every man loves
himself beyond all others, yet he holds his own opinion of himself in
less esteem than the opinion of others. Yet, if a God or some wise
teacher came and ordered a man to conceive and design nothing which he
would not utter the moment it occurred to him, he would not abide the
ordeal for a single day. Thus we stand in greater awe of our
neighbours' opinion of us than we do of our own.

5. How can it be that the Gods, who have ordered all things well for
man's advantage, overlooked one thing only, to wit that some of the
best of mankind, who have held the closest relations with things
divine, and by pious works and holy ministry become intimate with the
Divinity, once dead, should arise no more, but be altogether
extinguished? If this be truly so, be well assured that, if it ought
to have been otherwise they would have made it otherwise. Had it been
right it would have been practicable; and had it been according to
Nature, Nature would have effected it. From its not being so, if it
really be not so, be persuaded that it ought not to have been. You see
that, in debating this matter, you are pleading a point of justice
with the Gods. Now we would not thus plead with the Gods were they not
perfectly good and just. And, if they are so, they have left nothing
unjustly and unreasonably neglected in their administration.

6. Essay even tasks that you despair of executing. The left hand,
which in other things is of little value for want of use, yet holds
the bridle more firmly than the right, for in this it has practice.

7. Consider how death ought to find you, both as to body and as to
soul. Think of the shortness of life, of the eternities before and
after, and of the infirmity of all material things.

8. Contemplate the fundamental causes stripped of disguises. Think
what pain is, what pleasure is, what death, and what fame. Consider
how many are themselves the causes of all the disquiet that they
suffer; how no man may be hindered by another; how all is matter of
opinion.

9. In the use of principles we should be like the pugilist rather than
the swordsman. For when the latter drops the sword which he uses he is
undone. But the former has his hand always by him and needs but to
wield it.

10. Consider well the nature of things, distinguishing between matter,
cause, and purpose.

11. What a glorious power is given to man, never to do any action of
which God will not approve, and to welcome whatever God appoints for
him!

12. As to what happens in the course of nature, the Gods are not to be
blamed. They never do wrong, willingly or unwillingly. Neither are men
to be blamed, for they do no wrong willingly. There is therefore none
to blame.

13. How ridiculous, and how like a foreigner, is he who is surprised
at anything which happens in life!

14. There is either a fatal necessity, an unalterable order, or a
placable Providence, or a blind confusion without a governor. If there
be an unalterable necessity, why strive against it? If there be a
Providence admitting of propitiation, make yourself worthy of the
divine aid. If there be an ungoverned confusion, be comforted; seeing
that in this tempest you have within yourself a guiding
intelligence. And, if the wave should carry you away, let it carry
away the carcase and the animal life, for the intellectual part of you
it will not carry away.

15. If the light of a lamp shine and lose not its radiance until it be
extinguished, shall truth, justice, and temperance be extinguished in
you before your own extinction.

16. When you have the impression that a man has sinned, say to
yourself: ``How do I know that this is sin?'' And, if he has sinned,
consider that he stands self-condemned: and thus, as it were, has torn
his own face.

He that would wish the wicked not to sin is like one who would have
the fig tree not have juice in its figs, would have infants not cry,
horses not neigh, and other inevitable things not happen. What shall
the wicked man do, having a wicked disposition? If you are so keen,
cure it.

17. If a thing be not becoming, do it not; if not true, say it not.

18. Endeavour always to see in everything what it is that causes your
impression; and unfold it by distinguishing the cause, the matter, the
relation to other things, and the period within which it must cease to
exist.

19. Perceive at last that there is within you something better and
more divine than the immediate cause of your sensations of pleasure
and pain; something, in short, beyond the strings which move the
puppet. What is now my thought? Is it fear? Suspicion? Lust? Or any
such passion?

20. In the first place, let nothing be done at random or without an
object. In the second let your object never be other than the common
good.

21. Yet a little, and you shall be no more; nor shall any of these
things remain which you now behold, nor any of those who are now
living. It is the nature of all things to change, to turn, and to
corrupt; in order that other things may, in their course, spring out
of them.

22. Reflect that everything is matter of opinion; and opinion rests
with yourself, suppress then your opinion, what time you will, and
like one who has doubled the cape and reached the bay, you will have
calm and stillness everywhere, never a wave.

23. Any one natural operation, ending at its proper time, suffers no
ill by ceasing, nor does the agent therein suffer any ill by its thus
ceasing. In like manner, as to the whole series of actions which is
life, if it ends in its season it suffers no ill by ceasing, nor is he
who thus completes his series, in any evil case. The season and the
term are assigned by Nature; sometimes even by your own nature, as in
old age; but always by the nature of the whole, by the interchange of
whose parts the Universe still remains fresh and in its bloom. Now,
that is always good and seasonable which is advantageous to the nature
of the whole. Wherefore the ceasing of life cannot be evil to the
individual. There is no turpitude in it, since it is beyond our power,
and contains nothing contrary to the common advantage. Nay, it is
good, since it is seasonable and advantageous to the whole, and,
congruent with the order of the Universe. Thus, too, he is led by God
who goes the same way with God, and that by like inclination.

24. Have these three thoughts always at hand: First, as to your
action, do nothing inconsiderately, or otherwise than justice herself
would have acted. As for external events, they either happen by chance
or by providence; now, no man should quarrel with chance or censure
providence. Second, examine what each thing is, from its seed to its
quickening; and from its quickening to its death; of what materials it
is composed, and into what it will be resolved. Third, reflect that
could you be raised on high, and from thence behold all human affairs,
you would discern their great variety, conscious at the same time of
the crowds of serial and etherial inhabitants around us; but were you
so raised ever so often, you would always see the same things, all
uniform and of brief duration. Can we set our pride on such matters?

25. ``Cast away opinion, and you are saved.'' Who then hinders you from
casting it away?

26. When you fret at anything, you have forgotten that all happens in
accordance with the nature of the Universe, and that the wrong done
was another's. This, too, that whatever happens has happened, and will
happen, and is now happening everywhere. You have also forgotten how
great is the bond between any man and all the human race, a bond not
of blood and seed, but of common intelligence. You have forgotten that
the intelligence of every man is divine, and an efflux from God; also
that no man is proprietor of anything: his children, his body, his
very life are given of God. You have forgotten, too, that everything is
matter of opinion; and that it is the present moment only that one can
live or lose.

27. Bring to frequent recollection those who have grieved about
anything overmuch, those who have been pre-eminent in the extreme of
glory or misfortune, in feuds or other circumstances of fate. Then
stop and ask, Where are they all now? Smoke and ashes, and an old
tale; or perhaps not even a tale. Pass them all in review: Fabius
Catullinus in the country, Lucius Lupus in his gardens, Stertinius at
Baiae, Tiberius at Capreae, Velius Rufus, and, in fine, all eminence
attended with the high regard of men. How cheap is all that is so
eagerly pursued? And how much better does it become a philosopher to
show himself, in the part of the material world allotted to him, just,
temperate, and obedient to the Gods; and this with simplicity; for
most intolerable of all is the pride of false humility.

28. To those who ask, ``Where have you seen the Gods, and how assured
yourself of their existence, that you worship them?'' make this reply:
First, they are visible, even to the eye. Again, my own soul I cannot
see, and yet I reverence it. Thus, too, as regards the Gods, I
continually feel their power; and so I know that they exist, and I
worship them.

29. The safety of life is to see the whole nature of everything, and
to discern the matter and the form of its constitution; also to do
justice with all your heart, and to speak the truth. What remains but
to enjoy life, adding one good to an another, so as not to lose the
smallest interval?

30. There is but one light of the sun, although it be scattered upon
walls and hills, and a myriad other objects. There is but one common
substance, although it be divided among ten thousand bodies having as
many different qualities.  There is but one soul, though it be
distributed among countless different natures and individual
forms. There is but one intelligent spirit, though it may seem to be
divided. The other parts of these individuals of which we have spoken,
such as breath and matter, are void of perception and of mutual
affection; yet even they are held together by the intelligent spirit
and gravitate together. But intelligence has a special tendency to its
kind, and unites therewith, and the community of feeling is not
broken.

31. What do you desire? To live on? Or is it to feel or to desire? To
grow and to decay again? To speak or think? Which of all these seems
worthy to be desired? And, if each and all of them is despicable,
proceed to the last that remains, to follow reason and God. Now, it is
repugnant to reverence for reason and for God to grieve at the loss by
death of these other despicable things.

32. How small a part of the boundless immensity of the ages is
allotted to each of us, and presently that will vanish in eternity!
How little is ours of the universal substance; how little of the
universal spirit!  On what a little clod of the whole earth do we
creep! Considering all this, reckon nothing great except to act as
your nature leads you, and to endure what universal Nature brings to
pass.

33. How is it with your ruling part? On this all depends. All other
things, within or without our control, are but corpses, dust, and
smoke.

34. This most of all must rouse you to despise death: That even those
who held pleasure to be good and pain to be evil nevertheless despised
it.

35. To him who holds that alone to be good which comes in proper
season, who cares not whether he has acted oftener or less often
according to right reason; to whom it makes no difference whether he
behold the universe for a longer time or a shorter--for this man
death also has no terror.

36. You have lived, O man, as a citizen of this great city; of what
consequence to you whether for five years or for three? What comes by
law is fair to all. Where then is the calamity, if you are sent out of
the city, by no tyrant or unjust judge, but Nature herself who at
first introduced you, just as the praetor who engaged the actor again
dismisses him from the stage? ``But,'' say you, ``I have not spoken my
five acts, but only three.'' True, but in life three acts make up the
play. For he sets the end who was responsible for its composition at
the first, and for its present dissolution. You are responsible for
neither. Depart then graciously; for he who dismisses you is gracious.

\begin{center}THE END.\end{center}

\backmatter

\end{document}



%% 
%% End of the Project Gutenberg EBook of The Meditations of the Emperor Marcus
%% Aurelius Antoninus, by Marcus Aurelius
%% 
%% *** END OF THIS PROJECT GUTENBERG EBOOK MEDITATIONS OF MARCUS AURELIUS ANTONINUS ***
%% 
%% ***** This file should be named 55317.txt or 55317.zip *****
%% This and all associated files of various formats will be found in:
%%         http://www.gutenberg.org/5/5/3/1/55317/
%% 
%% Produced by E.H.N.
%% Updated editions will replace the previous one--the old editions will
%% be renamed.
%% 
%% Creating the works from print editions not protected by U.S. copyright
%% law means that no one owns a United States copyright in these works,
%% so the Foundation (and you!) can copy and distribute it in the United
%% States without permission and without paying copyright
%% royalties. Special rules, set forth in the General Terms of Use part
%% of this license, apply to copying and distributing Project
%% Gutenberg-tm electronic works to protect the PROJECT GUTENBERG-tm
%% concept and trademark. Project Gutenberg is a registered trademark,
%% and may not be used if you charge for the eBooks, unless you receive
%% specific permission. If you do not charge anything for copies of this
%% eBook, complying with the rules is very easy. You may use this eBook
%% for nearly any purpose such as creation of derivative works, reports,
%% performances and research. They may be modified and printed and given
%% away--you may do practically ANYTHING in the United States with eBooks
%% not protected by U.S. copyright law. Redistribution is subject to the
%% trademark license, especially commercial redistribution.
%% 
%% START: FULL LICENSE
%% 
%% THE FULL PROJECT GUTENBERG LICENSE
%% PLEASE READ THIS BEFORE YOU DISTRIBUTE OR USE THIS WORK
%% 
%% To protect the Project Gutenberg-tm mission of promoting the free
%% distribution of electronic works, by using or distributing this work
%% (or any other work associated in any way with the phrase "Project
%% Gutenberg"), you agree to comply with all the terms of the Full
%% Project Gutenberg-tm License available with this file or online at
%% www.gutenberg.org/license.
%% 
%% Section 1. General Terms of Use and Redistributing Project
%% Gutenberg-tm electronic works
%% 
%% 1.A. By reading or using any part of this Project Gutenberg-tm
%% electronic work, you indicate that you have read, understand, agree to
%% and accept all the terms of this license and intellectual property
%% (trademark/copyright) agreement. If you do not agree to abide by all
%% the terms of this agreement, you must cease using and return or
%% destroy all copies of Project Gutenberg-tm electronic works in your
%% possession. If you paid a fee for obtaining a copy of or access to a
%% Project Gutenberg-tm electronic work and you do not agree to be bound
%% by the terms of this agreement, you may obtain a refund from the
%% person or entity to whom you paid the fee as set forth in paragraph
%% 1.E.8.
%% 
%% 1.B. "Project Gutenberg" is a registered trademark. It may only be
%% used on or associated in any way with an electronic work by people who
%% agree to be bound by the terms of this agreement. There are a few
%% things that you can do with most Project Gutenberg-tm electronic works
%% even without complying with the full terms of this agreement. See
%% paragraph 1.C below. There are a lot of things you can do with Project
%% Gutenberg-tm electronic works if you follow the terms of this
%% agreement and help preserve free future access to Project Gutenberg-tm
%% electronic works. See paragraph 1.E below.
%% 
%% 1.C. The Project Gutenberg Literary Archive Foundation ("the
%% Foundation" or PGLAF), owns a compilation copyright in the collection
%% of Project Gutenberg-tm electronic works. Nearly all the individual
%% works in the collection are in the public domain in the United
%% States. If an individual work is unprotected by copyright law in the
%% United States and you are located in the United States, we do not
%% claim a right to prevent you from copying, distributing, performing,
%% displaying or creating derivative works based on the work as long as
%% all references to Project Gutenberg are removed. Of course, we hope
%% that you will support the Project Gutenberg-tm mission of promoting
%% free access to electronic works by freely sharing Project Gutenberg-tm
%% works in compliance with the terms of this agreement for keeping the
%% Project Gutenberg-tm name associated with the work. You can easily
%% comply with the terms of this agreement by keeping this work in the
%% same format with its attached full Project Gutenberg-tm License when
%% you share it without charge with others.
%% 
%% 1.D. The copyright laws of the place where you are located also govern
%% what you can do with this work. Copyright laws in most countries are
%% in a constant state of change. If you are outside the United States,
%% check the laws of your country in addition to the terms of this
%% agreement before downloading, copying, displaying, performing,
%% distributing or creating derivative works based on this work or any
%% other Project Gutenberg-tm work. The Foundation makes no
%% representations concerning the copyright status of any work in any
%% country outside the United States.
%% 
%% 1.E. Unless you have removed all references to Project Gutenberg:
%% 
%% 1.E.1. The following sentence, with active links to, or other
%% immediate access to, the full Project Gutenberg-tm License must appear
%% prominently whenever any copy of a Project Gutenberg-tm work (any work
%% on which the phrase "Project Gutenberg" appears, or with which the
%% phrase "Project Gutenberg" is associated) is accessed, displayed,
%% performed, viewed, copied or distributed:
%% 
%%   This eBook is for the use of anyone anywhere in the United States and
%%   most other parts of the world at no cost and with almost no
%%   restrictions whatsoever. You may copy it, give it away or re-use it
%%   under the terms of the Project Gutenberg License included with this
%%   eBook or online at www.gutenberg.org. If you are not located in the
%%   United States, you'll have to check the laws of the country where you
%%   are located before using this ebook.
%% 
%% 1.E.2. If an individual Project Gutenberg-tm electronic work is
%% derived from texts not protected by U.S. copyright law (does not
%% contain a notice indicating that it is posted with permission of the
%% copyright holder), the work can be copied and distributed to anyone in
%% the United States without paying any fees or charges. If you are
%% redistributing or providing access to a work with the phrase "Project
%% Gutenberg" associated with or appearing on the work, you must comply
%% either with the requirements of paragraphs 1.E.1 through 1.E.7 or
%% obtain permission for the use of the work and the Project Gutenberg-tm
%% trademark as set forth in paragraphs 1.E.8 or 1.E.9.
%% 
%% 1.E.3. If an individual Project Gutenberg-tm electronic work is posted
%% with the permission of the copyright holder, your use and distribution
%% must comply with both paragraphs 1.E.1 through 1.E.7 and any
%% additional terms imposed by the copyright holder. Additional terms
%% will be linked to the Project Gutenberg-tm License for all works
%% posted with the permission of the copyright holder found at the
%% beginning of this work.
%% 
%% 1.E.4. Do not unlink or detach or remove the full Project Gutenberg-tm
%% License terms from this work, or any files containing a part of this
%% work or any other work associated with Project Gutenberg-tm.
%% 
%% 1.E.5. Do not copy, display, perform, distribute or redistribute this
%% electronic work, or any part of this electronic work, without
%% prominently displaying the sentence set forth in paragraph 1.E.1 with
%% active links or immediate access to the full terms of the Project
%% Gutenberg-tm License.
%% 
%% 1.E.6. You may convert to and distribute this work in any binary,
%% compressed, marked up, nonproprietary or proprietary form, including
%% any word processing or hypertext form. However, if you provide access
%% to or distribute copies of a Project Gutenberg-tm work in a format
%% other than "Plain Vanilla ASCII" or other format used in the official
%% version posted on the official Project Gutenberg-tm web site
%% (www.gutenberg.org), you must, at no additional cost, fee or expense
%% to the user, provide a copy, a means of exporting a copy, or a means
%% of obtaining a copy upon request, of the work in its original "Plain
%% Vanilla ASCII" or other form. Any alternate format must include the
%% full Project Gutenberg-tm License as specified in paragraph 1.E.1.
%% 
%% 1.E.7. Do not charge a fee for access to, viewing, displaying,
%% performing, copying or distributing any Project Gutenberg-tm works
%% unless you comply with paragraph 1.E.8 or 1.E.9.
%% 
%% 1.E.8. You may charge a reasonable fee for copies of or providing
%% access to or distributing Project Gutenberg-tm electronic works
%% provided that
%% 
%% * You pay a royalty fee of 20% of the gross profits you derive from
%%   the use of Project Gutenberg-tm works calculated using the method
%%   you already use to calculate your applicable taxes. The fee is owed
%%   to the owner of the Project Gutenberg-tm trademark, but he has
%%   agreed to donate royalties under this paragraph to the Project
%%   Gutenberg Literary Archive Foundation. Royalty payments must be paid
%%   within 60 days following each date on which you prepare (or are
%%   legally required to prepare) your periodic tax returns. Royalty
%%   payments should be clearly marked as such and sent to the Project
%%   Gutenberg Literary Archive Foundation at the address specified in
%%   Section 4, "Information about donations to the Project Gutenberg
%%   Literary Archive Foundation."
%% 
%% * You provide a full refund of any money paid by a user who notifies
%%   you in writing (or by e-mail) within 30 days of receipt that s/he
%%   does not agree to the terms of the full Project Gutenberg-tm
%%   License. You must require such a user to return or destroy all
%%   copies of the works possessed in a physical medium and discontinue
%%   all use of and all access to other copies of Project Gutenberg-tm
%%   works.
%% 
%% * You provide, in accordance with paragraph 1.F.3, a full refund of
%%   any money paid for a work or a replacement copy, if a defect in the
%%   electronic work is discovered and reported to you within 90 days of
%%   receipt of the work.
%% 
%% * You comply with all other terms of this agreement for free
%%   distribution of Project Gutenberg-tm works.
%% 
%% 1.E.9. If you wish to charge a fee or distribute a Project
%% Gutenberg-tm electronic work or group of works on different terms than
%% are set forth in this agreement, you must obtain permission in writing
%% from both the Project Gutenberg Literary Archive Foundation and The
%% Project Gutenberg Trademark LLC, the owner of the Project Gutenberg-tm
%% trademark. Contact the Foundation as set forth in Section 3 below.
%% 
%% 1.F.
%% 
%% 1.F.1. Project Gutenberg volunteers and employees expend considerable
%% effort to identify, do copyright research on, transcribe and proofread
%% works not protected by U.S. copyright law in creating the Project
%% Gutenberg-tm collection. Despite these efforts, Project Gutenberg-tm
%% electronic works, and the medium on which they may be stored, may
%% contain "Defects," such as, but not limited to, incomplete, inaccurate
%% or corrupt data, transcription errors, a copyright or other
%% intellectual property infringement, a defective or damaged disk or
%% other medium, a computer virus, or computer codes that damage or
%% cannot be read by your equipment.
%% 
%% 1.F.2. LIMITED WARRANTY, DISCLAIMER OF DAMAGES - Except for the "Right
%% of Replacement or Refund" described in paragraph 1.F.3, the Project
%% Gutenberg Literary Archive Foundation, the owner of the Project
%% Gutenberg-tm trademark, and any other party distributing a Project
%% Gutenberg-tm electronic work under this agreement, disclaim all
%% liability to you for damages, costs and expenses, including legal
%% fees. YOU AGREE THAT YOU HAVE NO REMEDIES FOR NEGLIGENCE, STRICT
%% LIABILITY, BREACH OF WARRANTY OR BREACH OF CONTRACT EXCEPT THOSE
%% PROVIDED IN PARAGRAPH 1.F.3. YOU AGREE THAT THE FOUNDATION, THE
%% TRADEMARK OWNER, AND ANY DISTRIBUTOR UNDER THIS AGREEMENT WILL NOT BE
%% LIABLE TO YOU FOR ACTUAL, DIRECT, INDIRECT, CONSEQUENTIAL, PUNITIVE OR
%% INCIDENTAL DAMAGES EVEN IF YOU GIVE NOTICE OF THE POSSIBILITY OF SUCH
%% DAMAGE.
%% 
%% 1.F.3. LIMITED RIGHT OF REPLACEMENT OR REFUND - If you discover a
%% defect in this electronic work within 90 days of receiving it, you can
%% receive a refund of the money (if any) you paid for it by sending a
%% written explanation to the person you received the work from. If you
%% received the work on a physical medium, you must return the medium
%% with your written explanation. The person or entity that provided you
%% with the defective work may elect to provide a replacement copy in
%% lieu of a refund. If you received the work electronically, the person
%% or entity providing it to you may choose to give you a second
%% opportunity to receive the work electronically in lieu of a refund. If
%% the second copy is also defective, you may demand a refund in writing
%% without further opportunities to fix the problem.
%% 
%% 1.F.4. Except for the limited right of replacement or refund set forth
%% in paragraph 1.F.3, this work is provided to you 'AS-IS', WITH NO
%% OTHER WARRANTIES OF ANY KIND, EXPRESS OR IMPLIED, INCLUDING BUT NOT
%% LIMITED TO WARRANTIES OF MERCHANTABILITY OR FITNESS FOR ANY PURPOSE.
%% 
%% 1.F.5. Some states do not allow disclaimers of certain implied
%% warranties or the exclusion or limitation of certain types of
%% damages. If any disclaimer or limitation set forth in this agreement
%% violates the law of the state applicable to this agreement, the
%% agreement shall be interpreted to make the maximum disclaimer or
%% limitation permitted by the applicable state law. The invalidity or
%% unenforceability of any provision of this agreement shall not void the
%% remaining provisions.
%% 
%% 1.F.6. INDEMNITY - You agree to indemnify and hold the Foundation, the
%% trademark owner, any agent or employee of the Foundation, anyone
%% providing copies of Project Gutenberg-tm electronic works in
%% accordance with this agreement, and any volunteers associated with the
%% production, promotion and distribution of Project Gutenberg-tm
%% electronic works, harmless from all liability, costs and expenses,
%% including legal fees, that arise directly or indirectly from any of
%% the following which you do or cause to occur: (a) distribution of this
%% or any Project Gutenberg-tm work, (b) alteration, modification, or
%% additions or deletions to any Project Gutenberg-tm work, and (c) any
%% Defect you cause.
%% 
%% Section 2. Information about the Mission of Project Gutenberg-tm
%% 
%% Project Gutenberg-tm is synonymous with the free distribution of
%% electronic works in formats readable by the widest variety of
%% computers including obsolete, old, middle-aged and new computers. It
%% exists because of the efforts of hundreds of volunteers and donations
%% from people in all walks of life.
%% 
%% Volunteers and financial support to provide volunteers with the
%% assistance they need are critical to reaching Project Gutenberg-tm's
%% goals and ensuring that the Project Gutenberg-tm collection will
%% remain freely available for generations to come. In 2001, the Project
%% Gutenberg Literary Archive Foundation was created to provide a secure
%% and permanent future for Project Gutenberg-tm and future
%% generations. To learn more about the Project Gutenberg Literary
%% Archive Foundation and how your efforts and donations can help, see
%% Sections 3 and 4 and the Foundation information page at
%% www.gutenberg.org
%% 
%% 
%% 
%% Section 3. Information about the Project Gutenberg Literary Archive Foundation
%% 
%% The Project Gutenberg Literary Archive Foundation is a non profit
%% 501(c)(3) educational corporation organized under the laws of the
%% state of Mississippi and granted tax exempt status by the Internal
%% Revenue Service. The Foundation's EIN or federal tax identification
%% number is 64-6221541. Contributions to the Project Gutenberg Literary
%% Archive Foundation are tax deductible to the full extent permitted by
%% U.S. federal laws and your state's laws.
%% 
%% The Foundation's principal office is in Fairbanks, Alaska, with the
%% mailing address: PO Box 750175, Fairbanks, AK 99775, but its
%% volunteers and employees are scattered throughout numerous
%% locations. Its business office is located at 809 North 1500 West, Salt
%% Lake City, UT 84116, (801) 596-1887. Email contact links and up to
%% date contact information can be found at the Foundation's web site and
%% official page at www.gutenberg.org/contact
%% 
%% For additional contact information:
%% 
%%     Dr. Gregory B. Newby
%%     Chief Executive and Director
%%     gbnewby@pglaf.org
%% 
%% Section 4. Information about Donations to the Project Gutenberg
%% Literary Archive Foundation
%% 
%% Project Gutenberg-tm depends upon and cannot survive without wide
%% spread public support and donations to carry out its mission of
%% increasing the number of public domain and licensed works that can be
%% freely distributed in machine readable form accessible by the widest
%% array of equipment including outdated equipment. Many small donations
%% ($1 to $5,000) are particularly important to maintaining tax exempt
%% status with the IRS.
%% 
%% The Foundation is committed to complying with the laws regulating
%% charities and charitable donations in all 50 states of the United
%% States. Compliance requirements are not uniform and it takes a
%% considerable effort, much paperwork and many fees to meet and keep up
%% with these requirements. We do not solicit donations in locations
%% where we have not received written confirmation of compliance. To SEND
%% DONATIONS or determine the status of compliance for any particular
%% state visit www.gutenberg.org/donate
%% 
%% While we cannot and do not solicit contributions from states where we
%% have not met the solicitation requirements, we know of no prohibition
%% against accepting unsolicited donations from donors in such states who
%% approach us with offers to donate.
%% 
%% International donations are gratefully accepted, but we cannot make
%% any statements concerning tax treatment of donations received from
%% outside the United States. U.S. laws alone swamp our small staff.
%% 
%% Please check the Project Gutenberg Web pages for current donation
%% methods and addresses. Donations are accepted in a number of other
%% ways including checks, online payments and credit card donations. To
%% donate, please visit: www.gutenberg.org/donate
%% 
%% Section 5. General Information About Project Gutenberg-tm electronic works.
%% 
%% Professor Michael S. Hart was the originator of the Project
%% Gutenberg-tm concept of a library of electronic works that could be
%% freely shared with anyone. For forty years, he produced and
%% distributed Project Gutenberg-tm eBooks with only a loose network of
%% volunteer support.
%% 
%% Project Gutenberg-tm eBooks are often created from several printed
%% editions, all of which are confirmed as not protected by copyright in
%% the U.S. unless a copyright notice is included. Thus, we do not
%% necessarily keep eBooks in compliance with any particular paper
%% edition.
%% 
%% Most people start at our Web site which has the main PG search
%% facility: www.gutenberg.org
%% 
%% This Web site includes information about Project Gutenberg-tm,
%% including how to make donations to the Project Gutenberg Literary
%% Archive Foundation, how to help produce our new eBooks, and how to
%% subscribe to our email newsletter to hear about new eBooks.
%% 
%% === 
